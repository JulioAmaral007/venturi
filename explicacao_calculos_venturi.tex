\documentclass[12pt,a4paper]{article}
\usepackage[utf8]{inputenc}
\usepackage[portuguese]{babel}
\usepackage{amsmath}
\usepackage{amsfonts}
\usepackage{amssymb}
\usepackage{physics}
\usepackage{geometry}
\usepackage{listings}
\usepackage{xcolor}
\usepackage{hyperref}

\geometry{margin=2.5cm}

\lstset{
    language=Python,
    basicstyle=\ttfamily\small,
    keywordstyle=\color{blue},
    commentstyle=\color{green!60!black},
    stringstyle=\color{red},
    numbers=left,
    numberstyle=\tiny\color{gray},
    stepnumber=1,
    numbersep=5pt,
    frame=single,
    breaklines=true,
    showstringspaces=false
}

\title{Explicação Detalhada dos Cálculos do Simulador de Venturi}
\author{Análise do Código \texttt{simulator.py}}
\date{\today}

\begin{document}

\maketitle

\tableofcontents
\newpage

\section{Introdução}

Este documento apresenta uma explicação detalhada de cada cálculo realizado na classe \texttt{VenturiSimulator}, implementada no arquivo \texttt{app\_modules/simulator.py}. O simulador implementa os cálculos necessários para analisar o comportamento de um medidor de vazão tipo Venturi, considerando tanto o caso ideal (sem atrito, sem perdas) quanto o caso real com perdas de carga e atrito.

\subsection{Conceitos Fundamentais}

Um medidor de Venturi é um dispositivo utilizado para medir a vazão de um fluido em um tubo. Ele consiste em:
\begin{itemize}
    \item Uma seção de entrada com diâmetro $D_1$
    \item Uma garganta (seção estreita) com diâmetro $D_2 < D_1$
    \item Um difusor de saída que reconduz o fluido ao diâmetro original
\end{itemize}

O princípio de funcionamento baseia-se na equação de Bernoulli e na conservação de massa, onde a diferença de pressão entre a entrada e a garganta está relacionada com a vazão do fluido.

Este trabalho contempla os seguintes conteúdos fundamentais da mecânica dos fluidos:
\begin{itemize}
    \item Definição de fluido, formas de análise e descrição
    \item Propriedades dos fluidos
    \item Forças hidrostáticas e empuxo
    \item Equações Básicas na Forma Integral para um Volume de Controle
    \item Equação de Euler
    \item Equação de Bernoulli
    \item Escoamento em dutos
\end{itemize}

\section{Fundamentos Teóricos da Mecânica dos Fluidos}

\subsection{Definição de Fluido, Formas de Análise e Descrição}

\subsubsection{Definição de Fluido}

Um \textbf{fluido} é uma substância que se deforma continuamente sob a ação de uma tensão de cisalhamento, por menor que seja. Diferentemente dos sólidos, os fluidos não resistem a deformações e assumem a forma do recipiente que os contém.

Os fluidos podem ser classificados em:
\begin{itemize}
    \item \textbf{Líquidos:} Possuem volume definido, mas assumem a forma do recipiente. São praticamente incompressíveis.
    \item \textbf{Gases:} Não possuem volume nem forma definidos. São compressíveis.
\end{itemize}

No simulador de Venturi, trabalhamos com fluidos \textbf{incompressíveis} (líquidos), onde a densidade $\rho$ é constante.

\textbf{Aplicação no Código:} A densidade $\rho$ é um parâmetro de entrada do método \texttt{calcular()} (linha 9) e é utilizada em todos os cálculos de pressão e perdas de carga. Ver seção \ref{sec:estrutura} para detalhes.

\subsubsection{Formas de Análise}

Existem duas abordagens principais para descrever o escoamento de fluidos:

\paragraph{Abordagem de Lagrange:}
Descreve o movimento de partículas individuais do fluido ao longo do tempo. Cada partícula é identificada por sua posição inicial.

\paragraph{Abordagem de Euler:}
Descreve o campo de escoamento em pontos fixos do espaço. As propriedades do fluido (velocidade, pressão, densidade) são funções da posição e do tempo: $v(x,y,z,t)$, $P(x,y,z,t)$, $\rho(x,y,z,t)$.

No simulador de Venturi, utilizamos a \textbf{abordagem de Euler}, analisando o escoamento em seções fixas do medidor (entrada, garganta, saída).

\textbf{Aplicação no Código:} O simulador calcula propriedades (pressão, velocidade) em pontos fixos: seção 1 (entrada), seção 2 (garganta) e seção 3 (saída). As velocidades são calculadas nas linhas 31-32 do método \texttt{\_calcular\_desnivel\_de\_vazao()}.

\subsubsection{Descrição do Escoamento}

O escoamento pode ser classificado quanto a:
\begin{itemize}
    \item \textbf{Regime:} Permanente (propriedades não variam com o tempo) ou transiente
    \item \textbf{Compressibilidade:} Incompressível ($\rho = \text{constante}$) ou compressível
    \item \textbf{Viscosidade:} Viscoso (com atrito) ou invíscido (sem atrito)
    \item \textbf{Rotacionalidade:} Rotacional ou irrotacional
\end{itemize}

No medidor Venturi, consideramos escoamento \textbf{permanente, incompressível} e analisamos tanto o caso \textbf{invíscido} (modo ideal) quanto \textbf{viscoso} (modo realista).

\textbf{Aplicação no Código:} 
\begin{itemize}
    \item \textbf{Escoamento permanente:} As propriedades são calculadas uma única vez, assumindo regime permanente (método \texttt{calcular()}, linha 9).
    \item \textbf{Incompressível:} A densidade $\rho$ é constante em todos os cálculos.
    \item \textbf{Invíscido (modo ideal):} $k_{entrada} = 0$ e $f$ não é utilizado (linhas 34-37, 41-42).
    \item \textbf{Viscoso (modo realista):} $k_{entrada} = 0.04$ e perdas por atrito são calculadas (linhas 44-45, 65-68).
\end{itemize}

\subsection{Propriedades dos Fluidos}

As propriedades dos fluidos utilizadas no simulador são:

\subsubsection{Densidade ($\rho$)}

A densidade é a massa por unidade de volume:
\begin{equation}
\rho = \frac{m}{V} \quad \text{[kg/m³]}
\end{equation}

No simulador:
\begin{itemize}
    \item $\rho$: Densidade do fluido em escoamento (ex: água, $\rho = 1000$ kg/m³)
    \item $\rho_m$: Densidade do fluido manométrico (ex: mercúrio, $\rho_m = 13600$ kg/m³)
\end{itemize}

A diferença de densidades ($\rho_m - \rho$) é fundamental para o funcionamento do manômetro diferencial.

\textbf{Aplicação no Código:} A densidade $\rho$ é utilizada em:
\begin{itemize}
    \item Cálculo de pressões: linha 39, 52, 55, 57, 59
    \item Cálculo de perdas: linhas 45, 58, 61
    \item Cálculo do desnível manométrico: linha 63, onde $\Delta h = \frac{\Delta P}{(\rho_m - \rho) g}$
    \item Cálculo do número de Reynolds: linha 71, $Re = \frac{\rho v D}{\mu}$
\end{itemize}

\subsubsection{Viscosidade Dinâmica ($\mu$)}

A viscosidade dinâmica mede a resistência do fluido ao escoamento devido ao atrito interno:
\begin{equation}
\tau = \mu \frac{dv}{dy} \quad \text{[Pa·s]}
\end{equation}

onde $\tau$ é a tensão de cisalhamento e $dv/dy$ é o gradiente de velocidade.

No simulador, $\mu$ é utilizada para calcular o número de Reynolds, que caracteriza o regime de escoamento (laminar, transição ou turbulento).

\textbf{Aplicação no Código:} A viscosidade dinâmica $\mu$ é um parâmetro de entrada (linha 9) e é utilizada no método \texttt{calcular\_reynolds()} (linha 71) para calcular $Re = \frac{\rho v_1 D_1}{\mu}$. Ver seção \ref{sec:reynolds} para detalhes.

\subsubsection{Viscosidade Cinemática ($\nu$)}

Relacionada à viscosidade dinâmica pela densidade:
\begin{equation}
\nu = \frac{\mu}{\rho} \quad \text{[m²/s]}
\end{equation}

\subsubsection{Pressão ($P$)}

A pressão é a força por unidade de área exercida pelo fluido:
\begin{equation}
P = \frac{F}{A} \quad \text{[Pa = N/m²]}
\end{equation}

No medidor Venturi, medimos diferenças de pressão entre seções para determinar a vazão.

\textbf{Aplicação no Código:} As pressões são calculadas em:
\begin{itemize}
    \item $P_1$: Parâmetro de entrada (linha 9)
    \item $P_2$: Calculada na linha 39 usando Bernoulli
    \item $P_{2,fim}$: Calculada na linha 47 após considerar perdas
    \item $P_3$: Calculada na linha 52 (ideal) ou 58 (realista)
    \item $\Delta P$: Calculada na linha 49
\end{itemize}

\subsection{Forças Hidrostáticas e Empuxo}

\subsubsection{Forças Hidrostáticas}

Em um fluido em repouso, a pressão varia com a profundidade segundo:
\begin{equation}
P = P_0 + \rho g h
\end{equation}

onde:
\begin{itemize}
    \item $P_0$ é a pressão na superfície livre
    \item $\rho$ é a densidade do fluido
    \item $g$ é a aceleração da gravidade
    \item $h$ é a profundidade
\end{itemize}

\subsubsection{Empuxo}

O princípio de Arquimedes estabelece que um corpo imerso em um fluido sofre uma força de empuxo igual ao peso do fluido deslocado:
\begin{equation}
F_E = \rho g V_{deslocado}
\end{equation}

\subsubsection{Aplicação no Medidor Venturi}

No manômetro diferencial do Venturi, utilizamos o princípio hidrostático:
\begin{equation}
\Delta P = (\rho_m - \rho) g \Delta h
\end{equation}

onde $\Delta h$ é a diferença de altura no manômetro. Esta relação conecta a diferença de pressão dinâmica (devido ao escoamento) com a diferença de altura estática (no manômetro).

\textbf{Aplicação no Código:} O princípio hidrostático é aplicado na linha 63 do método \texttt{\_calcular\_desnivel\_de\_vazao()}:
\begin{equation}
\Delta h = \frac{\Delta P}{(\rho_m - \rho) g}
\end{equation}
Esta é a fórmula inversa da equação hidrostática, permitindo calcular o desnível manométrico a partir da diferença de pressão dinâmica.

\subsection{Equações Básicas na Forma Integral para um Volume de Controle}

As equações fundamentais da mecânica dos fluidos podem ser expressas na forma integral para um volume de controle:

\subsubsection{Conservação de Massa (Equação da Continuidade)}

Para um volume de controle fixo:
\begin{equation}
\frac{\partial}{\partial t} \iiint_{VC} \rho \, dV + \oiint_{SC} \rho \vec{v} \cdot \vec{n} \, dA = 0
\end{equation}

Para escoamento permanente e incompressível, simplifica-se para:
\begin{equation}
\oiint_{SC} \vec{v} \cdot \vec{n} \, dA = 0
\end{equation}

Aplicada ao medidor Venturi (seções 1 e 2):
\begin{equation}
v_1 A_1 = v_2 A_2 = Q = \text{constante}
\end{equation}

Esta é a equação da continuidade utilizada no simulador para calcular as velocidades.

\textbf{Aplicação no Código:} A equação da continuidade é aplicada diretamente nas linhas 31-32:
\begin{lstlisting}[firstnumber=31]
self.v1 = self.Q / self.A1
self.v2 = self.Q / self.A2
\end{lstlisting}
Estas linhas implementam $v_1 = Q/A_1$ e $v_2 = Q/A_2$, garantindo que $v_1 A_1 = v_2 A_2 = Q$ (conservação de massa). Ver seção \ref{sec:continuidade} para derivação completa.

\subsubsection{Conservação de Quantidade de Movimento}

Para um volume de controle:
\begin{equation}
\frac{\partial}{\partial t} \iiint_{VC} \rho \vec{v} \, dV + \oiint_{SC} \rho \vec{v} (\vec{v} \cdot \vec{n}) \, dA = \sum \vec{F}
\end{equation}

No medidor Venturi, esta equação explica as forças exercidas pelo fluido nas paredes do dispositivo.

\textbf{Aplicação no Código:} Embora não seja calculada explicitamente, a conservação de quantidade de movimento está implícita no cálculo das pressões. As forças de pressão nas paredes do Venturi são responsáveis pelas mudanças de direção do escoamento e pelas perdas de carga.

\subsubsection{Conservação de Energia (Primeira Lei da Termodinâmica)}

Para um volume de controle:
\begin{equation}
\frac{\partial}{\partial t} \iiint_{VC} \rho e \, dV + \oiint_{SC} \rho e (\vec{v} \cdot \vec{n}) \, dA = \dot{Q} - \dot{W}
\end{equation}

onde $e$ é a energia específica total (interna + cinética + potencial).

\subsection{Equação de Euler}

A equação de Euler descreve o movimento de um fluido invíscido (sem viscosidade). Para um fluido incompressível, em coordenadas cartesianas:
\begin{equation}
\frac{\partial \vec{v}}{\partial t} + (\vec{v} \cdot \nabla) \vec{v} = -\frac{1}{\rho} \nabla P + \vec{g}
\end{equation}

ou, na forma vetorial:
\begin{equation}
\frac{D\vec{v}}{Dt} = -\frac{1}{\rho} \nabla P + \vec{g}
\end{equation}

onde $\frac{D}{Dt}$ é a derivada material (derivada total).

\subsubsection{Integração da Equação de Euler - Equação de Bernoulli}

Integrando a equação de Euler ao longo de uma linha de corrente, para escoamento permanente e incompressível, obtemos a \textbf{equação de Bernoulli}:
\begin{equation}
P + \frac{1}{2}\rho v^2 + \rho g z = \text{constante}
\end{equation}

ou, entre dois pontos na mesma linha de corrente:
\begin{equation}
P_1 + \frac{1}{2}\rho v_1^2 + \rho g z_1 = P_2 + \frac{1}{2}\rho v_2^2 + \rho g z_2
\end{equation}

Esta é a base teórica para os cálculos de pressão no medidor Venturi.

\textbf{Aplicação no Código:} A equação de Euler integrada resulta na equação de Bernoulli, que é aplicada na linha 39 para calcular $P_2$:
\begin{lstlisting}[firstnumber=39]
self.P2 = self.P1 - 0.5 * self.rho * ((self.v2**2 * (1 + k_entrada)) - self.v1**2)
\end{lstlisting}
Esta linha implementa a forma modificada de Bernoulli: $P_2 = P_1 - \frac{1}{2}\rho \left[v_2^2(1 + k_{entrada}) - v_1^2\right]$. Ver seção \ref{sec:bernoulli} para derivação completa.

\subsection{Equação de Bernoulli}
\label{sec:bernoulli}

A equação de Bernoulli é uma das equações mais importantes da mecânica dos fluidos e é fundamental para o funcionamento do medidor Venturi.

\subsubsection{Forma Geral}

Para escoamento permanente, incompressível e invíscido ao longo de uma linha de corrente:
\begin{equation}
P + \frac{1}{2}\rho v^2 + \rho g z = \text{constante}
\end{equation}

onde cada termo representa:
\begin{itemize}
    \item $P$: Pressão estática (energia de pressão por unidade de volume)
    \item $\frac{1}{2}\rho v^2$: Pressão dinâmica (energia cinética por unidade de volume)
    \item $\rho g z$: Pressão hidrostática (energia potencial por unidade de volume)
\end{itemize}

\subsubsection{Forma Simplificada para Tubo Horizontal}

Quando $z_1 = z_2$ (tubo horizontal):
\begin{equation}
P_1 + \frac{1}{2}\rho v_1^2 = P_2 + \frac{1}{2}\rho v_2^2
\end{equation}

Esta é a forma utilizada no medidor Venturi, onde a diferença de pressão está relacionada à diferença de velocidades.

\textbf{Aplicação no Código:} A equação de Bernoulli simplificada é a base do cálculo de $P_2$ (linha 39). No modo ideal ($k_{entrada} = 0$), temos exatamente $P_1 + \frac{1}{2}\rho v_1^2 = P_2 + \frac{1}{2}\rho v_2^2$. Ver simulação Caso 1 (seção \ref{sec:simulacao1}) para exemplo numérico.

\subsubsection{Equação de Bernoulli Modificada (com Perdas)}

Para escoamento real (com perdas), a equação de Bernoulli é modificada:
\begin{equation}
P_1 + \frac{1}{2}\rho v_1^2 = P_2 + \frac{1}{2}\rho v_2^2 + \Delta P_{perdas}
\end{equation}

No simulador, as perdas são modeladas como:
\begin{equation}
\Delta P_{perdas} = k \cdot \frac{1}{2}\rho v^2
\end{equation}

onde $k$ é um coeficiente de perda que depende da geometria e das condições do escoamento.

\textbf{Aplicação no Código:} A equação de Bernoulli modificada é aplicada em:
\begin{itemize}
    \item \textbf{Linha 39:} Cálculo de $P_2$ com perda na entrada ($k_{entrada}$)
    \item \textbf{Linha 55:} Recuperação dinâmica: $\Delta P_{recuperacao} = \frac{1}{2}\rho(v_2^2 - v_1^2)$
    \item \textbf{Linha 57:} Perda no difusor: $\Delta P_{difusor} = K_{difusor} \cdot \frac{1}{2}\rho v_2^2$
    \item \textbf{Linha 58:} Pressão na saída: $P_3 = P_{2,fim} + \Delta P_{recuperacao} - \Delta P_{difusor}$
\end{itemize}
Ver seção \ref{sec:bernoulli} para derivação completa e seção \ref{sec:simulacoes} para exemplos numéricos.

\subsection{Escoamento em Dutos}

\subsubsection{Características do Escoamento em Dutos}

O escoamento em dutos (tubos) é caracterizado por:
\begin{itemize}
    \item Escoamento confinado (limitado pelas paredes)
    \item Perdas de carga devido ao atrito nas paredes
    \item Distribuição de velocidade não uniforme (perfil de velocidade)
    \item Regimes de escoamento: laminar, transição ou turbulento
\end{itemize}

\subsubsection{Perdas de Carga em Dutos}

\paragraph{Perdas por Atrito (Distribuídas)}

A equação de Darcy-Weisbach relaciona a perda de carga por atrito com as propriedades do escoamento:
\begin{equation}
h_f = f \frac{L}{D} \frac{v^2}{2g}
\end{equation}

onde:
\begin{itemize}
    \item $f$ é o fator de atrito de Darcy-Weisbach
    \item $L$ é o comprimento do trecho
    \item $D$ é o diâmetro hidráulico
    \item $v$ é a velocidade média
    \item $g$ é a aceleração da gravidade
\end{itemize}

Esta equação é utilizada no simulador para calcular as perdas na garganta do Venturi.

\textbf{Aplicação no Código:} A equação de Darcy-Weisbach é implementada no método \texttt{\_calcular\_perda\_carga\_garganta()} (linhas 65-68):
\begin{lstlisting}[firstnumber=65]
def _calcular_perda_carga_garganta(self):
    h_f_garganta = self.f * (self.L_garganta / self.D2) * (self.v2**2 / (2 * self.g))
    return h_f_garganta
\end{lstlisting}
Este método calcula $h_{f,garganta} = f \frac{L_{garganta}}{D_2} \frac{v_2^2}{2g}$, que é convertido para pressão na linha 45: $\Delta P_{garganta} = h_{f,garganta} \cdot \rho \cdot g$. Ver seção \ref{sec:perda_garganta} para detalhes.

\paragraph{Perdas Localizadas (Singulares)}

Perdas devido a mudanças de geometria (curvas, válvulas, contrações, expansões):
\begin{equation}
h_L = K \frac{v^2}{2g}
\end{equation}

onde $K$ é o coeficiente de perda localizada.

No medidor Venturi, temos perdas localizadas na:
\begin{itemize}
    \item Entrada (coeficiente $k_{entrada}$)
    \item Difusor (coeficiente $K_{difusor}$)
\end{itemize}

\textbf{Aplicação no Código:}
\begin{itemize}
    \item \textbf{Perda na entrada:} Coeficiente $k_{entrada}$ definido nas linhas 34-37 (0.0 no modo ideal, 0.04 no modo realista). Aplicado na linha 39 no cálculo de $P_2$ e calculada explicitamente na linha 59: $\Delta P_{entrada} = k_{entrada} \cdot \frac{1}{2}\rho v_2^2$.
    \item \textbf{Perda no difusor:} Coeficiente $K_{difusor}$ calculado no método \texttt{\_obter\_k\_difusor\_15\_graus()} (linhas 98-115) e aplicado na linha 57: $\Delta P_{difusor} = K_{difusor} \cdot \frac{1}{2}\rho v_2^2$.
\end{itemize}
Ver seção \ref{sec:difusor} para detalhes do cálculo de $K_{difusor}$.

\subsubsection{Número de Reynolds}

O número de Reynolds caracteriza o regime de escoamento:
\begin{equation}
Re = \frac{\rho v D}{\mu} = \frac{v D}{\nu}
\end{equation}

Classificação:
\begin{itemize}
    \item $Re < 2300$: Escoamento laminar
    \item $2300 < Re < 4000$: Escoamento de transição
    \item $Re > 4000$: Escoamento turbulento
\end{itemize}

No simulador, o número de Reynolds é calculado para caracterizar o regime de escoamento.

\textbf{Aplicação no Código:} O número de Reynolds é calculado no método \texttt{calcular\_reynolds()} (linhas 70-72):
\begin{lstlisting}[firstnumber=70]
def calcular_reynolds(self):
    Re = (self.rho * self.v1 * self.D1) / self.mu
    return Re
\end{lstlisting}
Este método implementa $Re = \frac{\rho v_1 D_1}{\mu}$, utilizando a velocidade e diâmetro da entrada. O resultado indica se o escoamento é laminar ($Re < 2300$), de transição ($2300 < Re < 4000$) ou turbulento ($Re > 4000$). Ver seção \ref{sec:reynolds} e simulações (seção \ref{sec:simulacoes}) para exemplos.

\subsubsection{Perfil de Velocidade}

Em escoamento laminar em duto circular, o perfil de velocidade é parabólico:
\begin{equation}
v(r) = v_{max} \left(1 - \frac{r^2}{R^2}\right)
\end{equation}

Em escoamento turbulento, o perfil é mais achatado próximo ao centro.

No medidor Venturi, utilizamos a \textbf{velocidade média} $v = Q/A$ para os cálculos.

\textbf{Aplicação no Código:} A velocidade média é calculada nas linhas 31-32 usando a equação da continuidade:
\begin{lstlisting}[firstnumber=31]
self.v1 = self.Q / self.A1
self.v2 = self.Q / self.A2
\end{lstlisting}
Estas velocidades médias são utilizadas em todos os cálculos subsequentes:
\begin{itemize}
    \item Cálculo de pressões (linha 39)
    \item Cálculo de perdas (linhas 44, 55, 57, 59)
    \item Cálculo do número de Reynolds (linha 71)
\end{itemize}
A utilização da velocidade média é uma aproximação válida para escoamento turbulento desenvolvido, que é o caso típico em medidores Venturi.

\section{Estrutura da Classe}
\label{sec:estrutura}

\subsection{Método \texttt{\_\_init\_\_}}

\begin{lstlisting}[firstnumber=6]
def __init__(self):
    self.g = 9.81
\end{lstlisting}

\textbf{Linha 7:} Inicialização da aceleração da gravidade
\begin{itemize}
    \item \texttt{self.g = 9.81}: Define a constante gravitacional $g = 9.81$ m/s², utilizada em todos os cálculos que envolvem energia potencial e perdas de carga expressas em altura.
    \item Esta constante é fundamental para converter pressões em alturas manométricas e vice-versa através da relação $P = \rho g h$.
\end{itemize}

\subsection{Método \texttt{calcular}}

Este é o método principal que recebe todos os parâmetros de entrada e realiza os cálculos principais.

\subsubsection{Parâmetros de Entrada}

\begin{lstlisting}[firstnumber=9]
def calcular(self, D1, D2, L_garganta, rho, rho_m, Q, delta_h, f, Cd, mode, mu, P1):
\end{lstlisting}

\begin{itemize}
    \item \texttt{D1}: Diâmetro da seção de entrada (m)
    \item \texttt{D2}: Diâmetro da garganta (m)
    \item \texttt{L\_garganta}: Comprimento da garganta (m)
    \item \texttt{rho}: Densidade do fluido em escoamento (kg/m³)
    \item \texttt{rho\_m}: Densidade do fluido manométrico (kg/m³)
    \item \texttt{Q}: Vazão volumétrica (m³/s)
    \item \texttt{delta\_h}: Diferença de altura no manômetro (m)
    \item \texttt{f}: Fator de atrito de Darcy-Weisbach (adimensional)
    \item \texttt{Cd}: Coeficiente de descarga (adimensional)
    \item \texttt{mode}: Modo de operação ('Ideal' ou 'Real')
    \item \texttt{mu}: Viscosidade dinâmica do fluido (Pa·s)
    \item \texttt{P1}: Pressão na seção de entrada (Pa)
\end{itemize}

\subsubsection{Atribuição de Variáveis}

\begin{lstlisting}[firstnumber=10]
self.D1 = D1
self.D2 = D2
self.L_garganta = L_garganta
self.rho = rho
self.rho_m = rho_m
self.Q = Q
self.delta_h = delta_h
self.f = f
self.Cd = Cd
self.mode = mode
self.mu = mu
self.P1 = P1
\end{lstlisting}

\textbf{Linhas 10-21:} Armazenamento dos parâmetros de entrada como atributos da classe para uso posterior nos cálculos.

\subsubsection{Cálculo das Áreas das Seções}

\begin{lstlisting}[firstnumber=23]
self.A1 = np.pi * (self.D1 / 2) ** 2
self.A2 = np.pi * (self.D2 / 2) ** 2
\end{lstlisting}

\textbf{Linha 23:} Cálculo da área da seção de entrada
\begin{equation}
A_1 = \pi \left(\frac{D_1}{2}\right)^2 = \frac{\pi D_1^2}{4}
\end{equation}
onde $A_1$ é a área da seção transversal circular de entrada.

\textbf{Linha 24:} Cálculo da área da garganta
\begin{equation}
A_2 = \pi \left(\frac{D_2}{2}\right)^2 = \frac{\pi D_2^2}{4}
\end{equation}
onde $A_2$ é a área da seção transversal circular da garganta.

\subsubsection{Chamadas de Métodos Auxiliares}

\begin{lstlisting}[firstnumber=26]
self._calcular_geometria_automatica()
self._calcular_desnivel_de_vazao()
\end{lstlisting}

\textbf{Linha 26:} Calcula automaticamente as dimensões geométricas do Venturi (comprimentos de entrada e saída).

\textbf{Linha 28:} Realiza o cálculo principal do desnível e das pressões ao longo do medidor.

\section{Cálculo do Desnível e Vazão}

\subsection{Método \texttt{\_calcular\_desnivel\_de\_vazao}}

Este método implementa a equação de Bernoulli modificada para calcular as pressões e perdas ao longo do medidor Venturi.

\subsubsection{Cálculo das Velocidades - Equação da Continuidade}
\label{sec:continuidade}

As velocidades nas seções 1 e 2 são calculadas utilizando a equação da continuidade. Vamos derivar e explicar esta equação fundamental:

\paragraph{Derivação da Equação da Continuidade}

A equação da continuidade é uma consequência direta do princípio de conservação de massa. Para um escoamento permanente e incompressível em um tubo, a massa que entra em uma seção deve ser igual à massa que sai.

\subparagraph{Passo 1: Conservação de Massa}

Para um volume de controle entre as seções 1 e 2, a taxa de massa que entra deve ser igual à taxa de massa que sai:
\begin{equation}
\dot{m}_1 = \dot{m}_2
\end{equation}

onde $\dot{m}$ é a vazão mássica (kg/s).

\subparagraph{Passo 2: Relação entre Vazão Mássica e Vazão Volumétrica}

A vazão mássica é relacionada à vazão volumétrica através da densidade:
\begin{equation}
\dot{m} = \rho Q = \rho v A
\end{equation}

onde:
\begin{itemize}
    \item $\rho$ é a densidade do fluido (kg/m³)
    \item $Q$ é a vazão volumétrica (m³/s)
    \item $v$ é a velocidade média na seção (m/s)
    \item $A$ é a área da seção transversal (m²)
\end{itemize}

\subparagraph{Passo 3: Aplicação da Conservação de Massa}

Aplicando a conservação de massa entre as seções 1 e 2:
\begin{equation}
\rho_1 v_1 A_1 = \rho_2 v_2 A_2
\end{equation}

\subparagraph{Passo 4: Simplificação para Fluido Incompressível}

Para um fluido incompressível, a densidade é constante ($\rho_1 = \rho_2 = \rho$), então:
\begin{equation}
\rho v_1 A_1 = \rho v_2 A_2
\end{equation}

Cancelando a densidade:
\begin{equation}
v_1 A_1 = v_2 A_2
\end{equation}

\subparagraph{Passo 5: Definição da Vazão Volumétrica}

A vazão volumétrica $Q$ é definida como:
\begin{equation}
Q = v A
\end{equation}

Como a vazão volumétrica é constante (conservação de massa), temos:
\begin{equation}
Q = v_1 A_1 = v_2 A_2 = \text{constante}
\end{equation}

\subparagraph{Passo 6: Isolamento das Velocidades}

Isolando as velocidades:
\begin{align}
v_1 &= \frac{Q}{A_1} \label{eq:v1} \\
v_2 &= \frac{Q}{A_2} \label{eq:v2}
\end{align}

Estas são as fórmulas utilizadas no código!

\paragraph{Aplicação no Código}

\begin{lstlisting}[firstnumber=31]
self.v1 = self.Q / self.A1
self.v2 = self.Q / self.A2
\end{lstlisting}

\textbf{Linha 31:} Velocidade média na seção de entrada
\begin{equation}
v_1 = \frac{Q}{A_1}
\end{equation}

\textbf{Linha 32:} Velocidade média na garganta
\begin{equation}
v_2 = \frac{Q}{A_2}
\end{equation}

\paragraph{Interpretação Física}

\begin{itemize}
    \item Como $A_2 < A_1$ (a garganta é mais estreita), temos que $v_2 > v_1$, conforme esperado pela conservação de massa.
    \item A velocidade aumenta na garganta para manter a vazão constante.
    \item A relação entre as velocidades é inversamente proporcional às áreas:
    \begin{equation}
    \frac{v_2}{v_1} = \frac{A_1}{A_2} = \left(\frac{D_1}{D_2}\right)^2
    \end{equation}
    \item Para um Venturi com $D_1 = 2D_2$, temos $v_2 = 4v_1$ (a velocidade quadruplica na garganta).
\end{itemize}

\paragraph{Exemplo Numérico}

Para o Caso 1 (modo ideal):
\begin{align*}
    A_1 &= \pi \left(\frac{0.1}{2}\right)^2 = 0.007854 \text{ m}^2 \\
    A_2 &= \pi \left(\frac{0.05}{2}\right)^2 = 0.001963 \text{ m}^2 \\
    Q &= 0.01 \text{ m}^3/\text{s} \\
    v_1 &= \frac{Q}{A_1} = \frac{0.01}{0.007854} = 1.2732 \text{ m/s} \\
    v_2 &= \frac{Q}{A_2} = \frac{0.01}{0.001963} = 5.0930 \text{ m/s}
\end{align*}

Verificando a conservação de massa:
\begin{align*}
    v_1 A_1 &= 1.2732 \times 0.007854 = 0.01 \text{ m}^3/\text{s} = Q \\
    v_2 A_2 &= 5.0930 \times 0.001963 = 0.01 \text{ m}^3/\text{s} = Q \\
    v_1 A_1 &= v_2 A_2 = Q \quad \checkmark
\end{align*}

A relação entre velocidades:
\begin{equation}
\frac{v_2}{v_1} = \frac{5.0930}{1.2732} = 4.0 = \left(\frac{D_1}{D_2}\right)^2 = \left(\frac{0.1}{0.05}\right)^2 = 4.0 \quad \checkmark
\end{equation}

\subsubsection{Coeficiente de Perda de Entrada}

\begin{lstlisting}[firstnumber=34]
if self.mode == 'Ideal':
    k_entrada = 0.0
else:
    k_entrada = 0.04
\end{lstlisting}

\textbf{Linhas 34-37:} Determinação do coeficiente de perda de carga na entrada
\begin{itemize}
    \item \textbf{Modo Ideal:} $k_{entrada} = 0$ (sem perdas, sem atrito)
    \item \textbf{Modo Real:} $k_{entrada} = 0.04$ (perda típica para entrada bem projetada, considerando atrito e outras dissipações)
\end{itemize}
O coeficiente $k$ relaciona a perda de pressão com a pressão dinâmica: $\Delta P = k \cdot \frac{1}{2}\rho v^2$. No modo ideal, este coeficiente é zero pois não há atrito nem outras perdas.

\subsubsection{Cálculo da Pressão na Garganta (antes das perdas)}

\begin{lstlisting}[firstnumber=39]
self.P2 = self.P1 - 0.5 * self.rho * ((self.v2**2 * (1 + k_entrada)) - self.v1**2)
\end{lstlisting}

\textbf{Linha 39:} Aplicação da equação de Bernoulli modificada para calcular $P_2$

\paragraph{Derivação Passo a Passo da Fórmula de $P_2$}

A derivação parte da equação de Bernoulli, que relaciona pressão, velocidade e altura em um escoamento de fluido. Vamos mostrar cada passo:

\subparagraph{Passo 1: Equação de Bernoulli Original}

A equação de Bernoulli para um fluido incompressível em escoamento permanente, sem atrito, entre dois pontos 1 e 2 na mesma linha de corrente é:
\begin{equation}
P_1 + \frac{1}{2}\rho v_1^2 + \rho g z_1 = P_2 + \frac{1}{2}\rho v_2^2 + \rho g z_2
\end{equation}

onde:
\begin{itemize}
    \item $P$ é a pressão estática
    \item $\rho$ é a densidade do fluido
    \item $v$ é a velocidade do fluido
    \item $g$ é a aceleração da gravidade
    \item $z$ é a altura em relação a um referencial
\end{itemize}

\subparagraph{Passo 2: Simplificação para Tubo Horizontal}

Como o medidor Venturi é tipicamente instalado horizontalmente, podemos assumir que $z_1 = z_2$ (mesma altura). Portanto, os termos de energia potencial se cancelam:
\begin{equation}
P_1 + \frac{1}{2}\rho v_1^2 = P_2 + \frac{1}{2}\rho v_2^2
\end{equation}

\subparagraph{Passo 3: Inclusão de Perdas na Entrada (Modo Realista)}

No caso real, há perdas de energia devido ao atrito e às mudanças de geometria. A perda na entrada pode ser modelada como uma fração da pressão dinâmica na garganta:
\begin{equation}
\Delta P_{entrada} = k_{entrada} \cdot \frac{1}{2}\rho v_2^2
\end{equation}

onde $k_{entrada}$ é o coeficiente de perda na entrada.

No modo ideal, $k_{entrada} = 0$ (sem perdas). No modo realista, $k_{entrada} = 0.04$ (perda típica).

A equação de Bernoulli modificada, incluindo a perda na entrada, fica:
\begin{equation}
P_1 + \frac{1}{2}\rho v_1^2 = P_2 + \frac{1}{2}\rho v_2^2 + \Delta P_{entrada}
\end{equation}

\subparagraph{Passo 4: Substituição da Perda na Entrada}

Substituindo a expressão da perda na entrada:
\begin{equation}
P_1 + \frac{1}{2}\rho v_1^2 = P_2 + \frac{1}{2}\rho v_2^2 + k_{entrada} \cdot \frac{1}{2}\rho v_2^2
\end{equation}

Fatorando o termo $\frac{1}{2}\rho v_2^2$:
\begin{equation}
P_1 + \frac{1}{2}\rho v_1^2 = P_2 + \frac{1}{2}\rho v_2^2 (1 + k_{entrada})
\end{equation}

\subparagraph{Passo 5: Isolamento de $P_2$}

Para isolar $P_2$, subtraímos $\frac{1}{2}\rho v_2^2 (1 + k_{entrada})$ de ambos os lados:
\begin{equation}
P_1 + \frac{1}{2}\rho v_1^2 - \frac{1}{2}\rho v_2^2 (1 + k_{entrada}) = P_2
\end{equation}

Reorganizando os termos:
\begin{equation}
P_2 = P_1 + \frac{1}{2}\rho v_1^2 - \frac{1}{2}\rho v_2^2 (1 + k_{entrada})
\end{equation}

\subparagraph{Passo 6: Fatoração Final}

Fatorando $\frac{1}{2}\rho$:
\begin{equation}
P_2 = P_1 - \frac{1}{2}\rho \left[v_2^2(1 + k_{entrada}) - v_1^2\right]
\end{equation}

Esta é a fórmula final utilizada no código!

\subparagraph{Interpretação Física}

A fórmula mostra que:
\begin{itemize}
    \item $P_2$ é menor que $P_1$ (o termo subtraído é positivo, pois $v_2 > v_1$)
    \item A diferença de pressão depende do quadrado das velocidades
    \item No modo ideal ($k_{entrada} = 0$), a fórmula se simplifica para:
    \begin{equation}
    P_2 = P_1 - \frac{1}{2}\rho (v_2^2 - v_1^2)
    \end{equation}
    \item No modo realista ($k_{entrada} > 0$), a perda adicional na entrada aumenta ainda mais a diferença de pressão
\end{itemize}

\paragraph{Exemplo Numérico (Modo Ideal)}

Para ilustrar com valores do Caso 1:
\begin{align*}
P_2 &= P_1 - \frac{1}{2}\rho \left[v_2^2(1 + k_{entrada}) - v_1^2\right] \\
P_2 &= 101325 - \frac{1}{2} \cdot 1000 \left[(5.0930)^2 \cdot (1 + 0.0) - (1.2732)^2\right] \\
P_2 &= 101325 - \frac{1}{2} \cdot 1000 \left[25.939 - 1.621\right] \\
P_2 &= 101325 - \frac{1}{2} \cdot 1000 \cdot 24.318 \\
P_2 &= 101325 - 12158.54 \\
P_2 &= 89166.46 \text{ Pa}
\end{align*}

Esta pressão representa a pressão teórica na garganta. No modo ideal, não há perdas por atrito, então esta é a pressão final. No modo realista, ainda será subtraída a perda por atrito na garganta.

\subsubsection{Cálculo das Perdas na Garganta}

\begin{lstlisting}[firstnumber=41]
if self.mode == 'Ideal':
    perda_garganta_Pa = 0.0
else:
    h_f_garganta = self._calcular_perda_carga_garganta()
    perda_garganta_Pa = h_f_garganta * self.rho * self.g
\end{lstlisting}

\textbf{Linhas 41-45:} Cálculo das perdas de carga na garganta

\begin{itemize}
    \item \textbf{Modo Ideal:} Sem perdas por atrito ($h_{f,garganta} = 0$). No modo ideal, não há atrito, portanto o fator de atrito $f$ não é utilizado e não há perdas de carga.
    \item \textbf{Modo Real:} 
    \begin{itemize}
        \item \textbf{Linha 44:} Calcula a perda de carga em altura (m) através do método auxiliar, que utiliza o fator de atrito $f$
        \item \textbf{Linha 45:} Converte a perda de altura em pressão (Pa):
        \begin{equation}
        \Delta P_{garganta} = h_{f,garganta} \cdot \rho \cdot g
        \end{equation}
    \end{itemize}
\end{itemize}

\subsubsection{Pressão Final na Garganta}

\begin{lstlisting}[firstnumber=47]
self.P2_fim = self.P2 - perda_garganta_Pa
\end{lstlisting}

\textbf{Linha 47:} Pressão real na saída da garganta
\begin{equation}
P_{2,fim} = P_2 - \Delta P_{garganta}
\end{equation}
Esta é a pressão efetiva após considerar as perdas por atrito na garganta. No modo ideal, $P_{2,fim} = P_2$ pois não há perdas.

\subsubsection{Diferença de Pressão}

\begin{lstlisting}[firstnumber=49]
self.delta_P = self.P1 - self.P2
\end{lstlisting}

\textbf{Linha 49:} Diferença de pressão entre entrada e garganta (antes das perdas na garganta)
\begin{equation}
\Delta P = P_1 - P_2
\end{equation}
Esta diferença é utilizada para calcular o desnível manométrico.

\subsubsection{Cálculo da Pressão na Saída}

\begin{lstlisting}[firstnumber=51]
if self.mode == 'Ideal':
    self.P3 = self.P1 
    self.h_L = 0.0
else:
    recuperacao_dinamica = 0.5 * self.rho * (self.v2**2 - self.v1**2)
    K_difusor = self._obter_k_difusor_15_graus()
    perda_difusor_Pa = K_difusor * (0.5 * self.rho * self.v2**2)
    self.P3 = self.P2_fim + recuperacao_dinamica - perda_difusor_Pa
    perda_entrada_Pa = k_entrada * (0.5 * self.rho * self.v2**2)
    perda_total_Pa = perda_entrada_Pa + perda_garganta_Pa + perda_difusor_Pa
    self.h_L = perda_total_Pa / (self.rho * self.g)
\end{lstlisting}

\textbf{Linhas 51-61:} Cálculo da pressão na seção de saída ($P_3$)

\paragraph{Modo Ideal (linhas 51-53):}
\begin{itemize}
    \item \textbf{Linha 52:} $P_3 = P_1$ (recuperação total de pressão, sem perdas por atrito ou outras dissipações)
    \item \textbf{Linha 53:} $h_L = 0$ (perda de carga total nula, pois não há atrito no modo ideal)
\end{itemize}

\paragraph{Modo Real (linhas 54-61):}

\textbf{Linha 55:} Recuperação dinâmica de pressão
\begin{equation}
\Delta P_{recuperacao} = \frac{1}{2}\rho(v_2^2 - v_1^2)
\end{equation}
Quando o fluido desacelera no difusor (de $v_2$ para $v_1$), parte da energia cinética é convertida em pressão estática.

\textbf{Linha 56:} Obtém o coeficiente de perda do difusor através de método auxiliar.

\textbf{Linha 57:} Perda de pressão no difusor
\begin{equation}
\Delta P_{difusor} = K_{difusor} \cdot \frac{1}{2}\rho v_2^2
\end{equation}
O coeficiente $K_{difusor}$ depende da geometria do difusor e da razão de áreas.

\textbf{Linha 58:} Pressão final na saída
\begin{equation}
P_3 = P_{2,fim} + \Delta P_{recuperacao} - \Delta P_{difusor}
\end{equation}
A pressão na saída é a pressão na garganta, mais a recuperação dinâmica, menos as perdas no difusor.

\textbf{Linha 59:} Perda de pressão na entrada
\begin{equation}
\Delta P_{entrada} = k_{entrada} \cdot \frac{1}{2}\rho v_2^2
\end{equation}

\textbf{Linha 60:} Perda de pressão total
\begin{equation}
\Delta P_{total} = \Delta P_{entrada} + \Delta P_{garganta} + \Delta P_{difusor}
\end{equation}

\textbf{Linha 61:} Perda de carga total em altura
\begin{equation}
h_L = \frac{\Delta P_{total}}{\rho g}
\end{equation}
Conversão da perda de pressão total para altura equivalente.

\subsubsection{Cálculo do Desnível Manométrico}

\begin{lstlisting}[firstnumber=63]
self.delta_h = self.delta_P / ((self.rho_m - self.rho) * self.g)
\end{lstlisting}

\textbf{Linha 63:} Cálculo do desnível no manômetro diferencial

A equação do manômetro diferencial relaciona a diferença de pressão com a diferença de altura:
\begin{equation}
\Delta P = (\rho_m - \rho) g \Delta h
\end{equation}

Rearranjando para isolar $\Delta h$:
\begin{equation}
\Delta h = \frac{\Delta P}{(\rho_m - \rho) g}
\end{equation}

onde:
\begin{itemize}
    \item $\rho_m$ é a densidade do fluido manométrico
    \item $\rho$ é a densidade do fluido em escoamento
    \item A diferença $(\rho_m - \rho)$ garante que o manômetro funcione corretamente
\end{itemize}

\section{Cálculo da Perda de Carga na Garganta}
\label{sec:perda_garganta}

\subsection{Método \texttt{\_calcular\_perda\_carga\_garganta}}

\textbf{Nota importante:} Este método é utilizado apenas no modo realista. No modo ideal, não há atrito e este método não é chamado.

\begin{lstlisting}[firstnumber=65]
def _calcular_perda_carga_garganta(self):
    h_f_garganta = self.f * (self.L_garganta / self.D2) * (self.v2**2 / (2 * self.g))
    
    return h_f_garganta
\end{lstlisting}

\textbf{Linha 66:} Aplicação da equação de Darcy-Weisbach

A perda de carga por atrito em um tubo é calculada pela equação de Darcy-Weisbach:
\begin{equation}
h_f = f \frac{L}{D} \frac{v^2}{2g}
\end{equation}

onde:
\begin{itemize}
    \item $f$ é o fator de atrito de Darcy-Weisbach (adimensional)
    \item $L$ é o comprimento do trecho (m)
    \item $D$ é o diâmetro hidráulico (m)
    \item $v$ é a velocidade média do fluido (m/s)
    \item $g$ é a aceleração da gravidade (m/s²)
\end{itemize}

No caso específico:
\begin{equation}
h_{f,garganta} = f \frac{L_{garganta}}{D_2} \frac{v_2^2}{2g}
\end{equation}

\textbf{Linha 68:} Retorna o valor calculado da perda de carga em metros.

\section{Cálculo do Número de Reynolds}
\label{sec:reynolds}

\subsection{Método \texttt{calcular\_reynolds}}

\begin{lstlisting}[firstnumber=70]
def calcular_reynolds(self):
    Re = (self.rho * self.v1 * self.D1) / self.mu
    return Re
\end{lstlisting}

\textbf{Linha 71:} Cálculo do número de Reynolds

O número de Reynolds é um parâmetro adimensional que caracteriza o regime de escoamento:
\begin{equation}
Re = \frac{\rho v D}{\mu} = \frac{v D}{\nu}
\end{equation}

onde:
\begin{itemize}
    \item $\rho$ é a densidade do fluido (kg/m³)
    \item $v$ é a velocidade característica (m/s) - utiliza-se $v_1$ da entrada
    \item $D$ é o diâmetro característico (m) - utiliza-se $D_1$ da entrada
    \item $\mu$ é a viscosidade dinâmica (Pa·s)
    \item $\nu = \mu/\rho$ é a viscosidade cinemática (m²/s)
\end{itemize}

O número de Reynolds indica:
\begin{itemize}
    \item $Re < 2300$: Escoamento laminar
    \item $2300 < Re < 4000$: Escoamento de transição
    \item $Re > 4000$: Escoamento turbulento
\end{itemize}

\textbf{Linha 72:} Retorna o número de Reynolds calculado.

\section{Cálculo da Perda Permanente}

\subsection{Método \texttt{\_obter\_perda\_permanente\_pct}}

\begin{lstlisting}[firstnumber=74]
def _obter_perda_permanente_pct(self):
    beta = self.D2 / self.D1

    perda_pct = 0.28 - (0.35 * beta) + (0.18 * beta**2)
    
    return max(0.10, min(perda_pct, 0.30))
\end{lstlisting}

\textbf{Linha 75:} Cálculo da razão de diâmetros
\begin{equation}
\beta = \frac{D_2}{D_1}
\end{equation}
A razão $\beta$ é um parâmetro geométrico fundamental do medidor Venturi, sempre menor que 1.

\textbf{Linha 77:} Cálculo da perda permanente percentual

A perda permanente é expressa como uma porcentagem da diferença de pressão total. A fórmula empírica utilizada é:
\begin{equation}
\text{perda\_pct} = 0.28 - 0.35\beta + 0.18\beta^2
\end{equation}

Esta é uma correlação empírica que relaciona a geometria do Venturi com a perda de pressão permanente.

\textbf{Linha 79:} Limitação do valor entre 10\% e 30\%
\begin{equation}
\text{perda\_pct} = \max(0.10, \min(\text{perda\_pct}, 0.30))
\end{equation}

Esta limitação garante que a perda permanente esteja dentro de uma faixa fisicamente razoável, independentemente do valor calculado pela fórmula empírica.

\section{Cálculo da Geometria Automática}

\subsection{Método \texttt{\_calcular\_geometria\_automatica}}

Este método calcula automaticamente os comprimentos dos cones de entrada e saída do Venturi, assumindo um ângulo de abertura de 15 graus.

\begin{lstlisting}[firstnumber=81]
def _calcular_geometria_automatica(self):
    angulo_graus = 15.0
    angulo_rad = np.radians(angulo_graus)
    
    delta_raio = (self.D1 - self.D2) / 2
    
    if delta_raio <= 0:
        self.L_entrada = 0
        self.L_saida = 0
    else:
        comprimento_cone = delta_raio / np.tan(angulo_rad / 2)
        
        self.L_entrada = comprimento_cone
        self.L_saida = comprimento_cone
    
    self.L = self.L_entrada + self.L_garganta + self.L_saida
\end{lstlisting}

\textbf{Linha 82:} Definição do ângulo de abertura
\begin{equation}
\theta = 15° \text{ (ângulo total do cone)}
\end{equation}
Este é um ângulo típico recomendado para minimizar perdas e evitar separação do escoamento.

\textbf{Linha 83:} Conversão para radianos
\begin{equation}
\theta_{rad} = \frac{\pi}{180} \cdot \theta_{graus}
\end{equation}
Necessário para cálculos trigonométricos em Python.

\textbf{Linha 85:} Cálculo da diferença de raio
\begin{equation}
\Delta r = \frac{D_1 - D_2}{2} = r_1 - r_2
\end{equation}
Representa a redução de raio do cone de entrada ou o aumento no cone de saída.

\textbf{Linhas 87-89:} Verificação de validade
\begin{itemize}
    \item Se $\Delta r \leq 0$, significa que $D_2 \geq D_1$, o que não é um Venturi válido
    \item Neste caso, os comprimentos são definidos como zero
\end{itemize}

\textbf{Linha 91:} Cálculo do comprimento do cone

A partir da geometria do cone, temos:
\begin{equation}
\tan\left(\frac{\theta}{2}\right) = \frac{\Delta r}{L_{cone}}
\end{equation}

Rearranjando:
\begin{equation}
L_{cone} = \frac{\Delta r}{\tan(\theta/2)}
\end{equation}

\textbf{Linhas 93-94:} Atribuição dos comprimentos
\begin{itemize}
    \item \textbf{Linha 93:} $L_{entrada} = L_{cone}$ (comprimento do cone convergente)
    \item \textbf{Linha 94:} $L_{saida} = L_{cone}$ (comprimento do cone divergente)
\end{itemize}

\textbf{Linha 96:} Comprimento total do Venturi
\begin{equation}
L_{total} = L_{entrada} + L_{garganta} + L_{saida}
\end{equation}

\section{Cálculo do Coeficiente de Perda do Difusor}
\label{sec:difusor}

\subsection{Método \texttt{\_obter\_k\_difusor\_15\_graus}}

Este método calcula o coeficiente de perda de carga do difusor ($K_{difusor}$) baseado na razão de áreas e em correlações empíricas.

\begin{lstlisting}[firstnumber=98]
def _obter_k_difusor_15_graus(self):

    AR = (self.D1 / self.D2) ** 2
 
    cp_ideal = 1.0 - (1.0 / AR**2)

    if AR < 1.2:
        cp_real = 1.4 * (AR - 1.0)
    elif AR > 4.0:
        cp_real = 0.64
    else:
        cp_real = (0.0394 * AR**3) - (0.3954 * AR**2) + (1.3095 * AR) - 0.7897
        
    cp_real = max(0.0, min(cp_real, cp_ideal))
    
    k_difusor = cp_ideal - cp_real
    
    return max(0.0, k_difusor)
\end{lstlisting}

\textbf{Linha 100:} Cálculo da razão de áreas
\begin{equation}
AR = \left(\frac{D_1}{D_2}\right)^2 = \frac{A_1}{A_2}
\end{equation}
A razão de áreas é sempre maior que 1, pois $D_1 > D_2$.

\textbf{Linha 102:} Coeficiente de pressão ideal
\begin{equation}
C_{p,ideal} = 1 - \frac{1}{AR^2}
\end{equation}

Este é o coeficiente de pressão teórico para um difusor ideal (sem perdas), baseado na equação de Bernoulli. Representa a recuperação máxima de pressão possível.

\textbf{Linhas 104-109:} Cálculo do coeficiente de pressão real

O coeficiente de pressão real depende da razão de áreas e é calculado através de correlações empíricas:

\paragraph{Se $AR < 1.2$ (linha 105):}
\begin{equation}
C_{p,real} = 1.4(AR - 1.0)
\end{equation}
Para razões de área pequenas, utiliza-se uma relação linear.

\paragraph{Se $AR > 4.0$ (linha 107):}
\begin{equation}
C_{p,real} = 0.64
\end{equation}
Para razões de área muito grandes, o coeficiente atinge um valor assintótico constante.

\paragraph{Se $1.2 \leq AR \leq 4.0$ (linha 109):}
\begin{equation}
C_{p,real} = 0.0394 AR^3 - 0.3954 AR^2 + 1.3095 AR - 0.7897
\end{equation}
Para valores intermediários, utiliza-se um polinômio de terceiro grau que interpola entre os regimes.

\textbf{Linha 111:} Limitação do coeficiente real
\begin{equation}
C_{p,real} = \max(0, \min(C_{p,real}, C_{p,ideal}))
\end{equation}
Garante que:
\begin{itemize}
    \item $C_{p,real} \geq 0$ (não pode ser negativo)
    \item $C_{p,real} \leq C_{p,ideal}$ (não pode exceder o valor ideal)
\end{itemize}

\textbf{Linha 113:} Cálculo do coeficiente de perda
\begin{equation}
K_{difusor} = C_{p,ideal} - C_{p,real}
\end{equation}

O coeficiente de perda representa a diferença entre a recuperação ideal e a real. Quanto maior $K_{difusor}$, maiores são as perdas no difusor.

\textbf{Linha 115:} Garantia de valor não-negativo
\begin{equation}
K_{difusor} = \max(0, K_{difusor})
\end{equation}
Assegura que o coeficiente de perda nunca seja negativo, o que seria fisicamente inconsistente.

\section{Simulações Numéricas}
\label{sec:simulacoes}

Esta seção apresenta simulações numéricas detalhadas para diferentes casos de estudo, ilustrando passo a passo cada cálculo realizado pelo simulador. As simulações demonstram a aplicação prática de todas as fórmulas explicadas anteriormente.

\subsection{Simulação: Caso 1 - Modo Ideal - Água}
\label{sec:simulacao1}

\subsubsection{Parâmetros de Entrada}

\begin{itemize}
    \item Modo de operação: \textbf{Ideal} (sem atrito, sem perdas)
    \item Diâmetro de entrada: $D_1 = 0.1$ m
    \item Diâmetro da garganta: $D_2 = 0.05$ m
    \item Comprimento da garganta: $L_{garganta} = 0.5$ m
    \item Fluido selecionado: Água (densidade $\rho$ e viscosidade $\mu$ calculadas automaticamente)
    \item Temperatura: $T = 20$ °C
    \item Pressão na entrada: $P_1 = 0$ Pa (manométricos, equivalente a pressão atmosférica)
    \item Densidade do fluido manométrico: $\rho_m = 13600$ kg/m³
    \item Variável de entrada: Vazão volumétrica $Q = 0.01$ m³/s
\end{itemize}

\textbf{Nota:} As propriedades do fluido (densidade $\rho$ e viscosidade dinâmica $\mu$) são calculadas automaticamente pela biblioteca \texttt{thermo} com base no fluido selecionado, temperatura e pressão. O fator de atrito $f$ é calculado automaticamente no modo Realista com base no material do tubo, número de Reynolds e rugosidade.

\subsubsection{Cálculos Intermediários}

\paragraph{Áreas das Seções:}
\begin{align*}
    A_1 &= \pi \left(\frac{D_1}{2}\right)^2 = \pi \left(\frac{0.1}{2}\right)^2 = 0.007854 \text{ m}^2 \\
    A_2 &= \pi \left(\frac{D_2}{2}\right)^2 = \pi \left(\frac{0.05}{2}\right)^2 = 0.001963 \text{ m}^2
\end{align*}

\paragraph{Velocidades - Aplicação da Equação da Continuidade:}

As velocidades são calculadas utilizando a equação da continuidade, que expressa a conservação de massa:

\subparagraph{Equação da Continuidade:}

Para um fluido incompressível em escoamento permanente, a vazão volumétrica é constante:
\begin{equation}
Q = v_1 A_1 = v_2 A_2 = \text{constante}
\end{equation}

Isolando as velocidades:
\begin{align}
v_1 &= \frac{Q}{A_1} \\
v_2 &= \frac{Q}{A_2}
\end{align}

\subparagraph{Cálculo Numérico:}
\begin{align*}
    v_1 &= \frac{Q}{A_1} = \frac{0.01}{0.007854} = 1.2732 \text{ m/s} \\
    v_2 &= \frac{Q}{A_2} = \frac{0.01}{0.001963} = 5.0930 \text{ m/s}
\end{align*}

\subparagraph{Verificação da Conservação de Massa:}
\begin{align*}
    v_1 A_1 &= 1.2732 \times 0.007854 = 0.01 \text{ m}^3/\text{s} = Q \\
    v_2 A_2 &= 5.0930 \times 0.001963 = 0.01 \text{ m}^3/\text{s} = Q \\
    v_1 A_1 &= v_2 A_2 = Q \quad \checkmark
\end{align*}

\subparagraph{Relação entre Velocidades:}
Como $A_2 < A_1$, a velocidade na garganta é maior:
\begin{equation}
\frac{v_2}{v_1} = \frac{5.0930}{1.2732} = 4.0 = \left(\frac{D_1}{D_2}\right)^2 = \left(\frac{0.1}{0.05}\right)^2 = 4.0
\end{equation}

A velocidade na garganta é 4 vezes maior que na entrada, pois a área é 4 vezes menor.

\paragraph{Coeficiente de Perda na Entrada:}
$k_{entrada} = 0.0$ (modo Ideal - sem perdas)

\paragraph{Pressão na Garganta - Derivação Passo a Passo:}

Para calcular a pressão na garganta $P_2$, partimos da equação de Bernoulli e derivamos a fórmula passo a passo:

\subparagraph{Passo 1: Equação de Bernoulli Original}

A equação de Bernoulli para um fluido incompressível entre os pontos 1 (entrada) e 2 (garganta) é:
\begin{equation}
P_1 + \frac{1}{2}\rho v_1^2 + \rho g z_1 = P_2 + \frac{1}{2}\rho v_2^2 + \rho g z_2
\end{equation}

\subparagraph{Passo 2: Simplificação para Tubo Horizontal}

Como o medidor Venturi é horizontal, $z_1 = z_2$, então os termos de energia potencial se cancelam:
\begin{equation}
P_1 + \frac{1}{2}\rho v_1^2 = P_2 + \frac{1}{2}\rho v_2^2
\end{equation}

\subparagraph{Passo 3: Inclusão de Perdas na Entrada}

No modo ideal, $k_{entrada} = 0$ (sem perdas). A equação permanece:
\begin{equation}
P_1 + \frac{1}{2}\rho v_1^2 = P_2 + \frac{1}{2}\rho v_2^2
\end{equation}

\subparagraph{Passo 4: Isolamento de $P_2$}

Subtraindo $\frac{1}{2}\rho v_2^2$ de ambos os lados:
\begin{equation}
P_2 = P_1 + \frac{1}{2}\rho v_1^2 - \frac{1}{2}\rho v_2^2
\end{equation}

\subparagraph{Passo 5: Fatoração Final}

Fatorando $\frac{1}{2}\rho$:
\begin{equation}
P_2 = P_1 - \frac{1}{2}\rho \left[v_2^2 - v_1^2\right]
\end{equation}

Como $k_{entrada} = 0$ no modo ideal, a fórmula geral $P_2 = P_1 - \frac{1}{2}\rho \left[v_2^2(1 + k_{entrada}) - v_1^2\right]$ se simplifica para a forma acima.

\subparagraph{Passo 6: Aplicação Numérica}

Substituindo os valores calculados:
\begin{align*}
    P_2 &= P_1 - \frac{1}{2}\rho \left[v_2^2 - v_1^2\right] \\
    P_2 &= 101325 - \frac{1}{2} \cdot 1000 \left[(5.0930)^2 - (1.2732)^2\right] \\
    P_2 &= 101325 - \frac{1}{2} \cdot 1000 \left[25.939 - 1.621\right] \\
    P_2 &= 101325 - \frac{1}{2} \cdot 1000 \cdot 24.318 \\
    P_2 &= 101325 - 12158.54 \\
    P_2 &= 89166.46 \text{ Pa}
\end{align*}

\textbf{Interpretação:} A pressão na garganta ($P_2 = 89166.46$ Pa) é menor que a pressão na entrada ($P_1 = 101325$ Pa) devido ao aumento da velocidade (de $v_1 = 1.2732$ m/s para $v_2 = 5.0930$ m/s), conforme o princípio de conservação de energia de Bernoulli.

\paragraph{Perdas na Garganta:}
Modo ideal: sem perdas por atrito ($\Delta P_{garganta} = 0$). O fator de atrito $f$ não é utilizado, pois não há atrito no modo ideal.

\paragraph{Diferença de Pressão:}
\begin{align*}
    \Delta P &= P_1 - P_2 = 101325 - 89166.46 = 12158.54 \text{ Pa}
\end{align*}

\paragraph{Pressão na Saída:}
Modo ideal: $P_3 = P_1 = 101325.00$ Pa (recuperação total de pressão, sem perdas por atrito ou outras dissipações)

\paragraph{Desnível Manométrico:}
\begin{align*}
    \Delta h &= \frac{\Delta P}{(\rho_m - \rho) g} = \frac{12158.54}{(13600 - 1000) \cdot 9.81} = 0.098365 \text{ m}
\end{align*}

\paragraph{Número de Reynolds:}
\begin{align*}
    Re &= \frac{\rho v_1 D_1}{\mu} = \frac{1000 \cdot 1.2732 \cdot 0.1}{0.001} = 127323.95
\end{align*}

\paragraph{Geometria do Venturi:}
\begin{align*}
    L_{entrada} &= 0.1899 \text{ m} \\
    L_{saida} &= 0.1899 \text{ m} \\
    L_{total} &= L_{entrada} + L_{garganta} + L_{saida} = 0.8798 \text{ m}
\end{align*}

\subsubsection{Resumo dos Resultados}

\begin{table}[h]
\centering
\begin{tabular}{l|c}
\hline
\textbf{Grandeza} & \textbf{Valor} \\
\hline
$A_1$ (m²) & 0.007854 \\
$A_2$ (m²) & 0.001963 \\
$v_1$ (m/s) & 1.2732 \\
$v_2$ (m/s) & 5.0930 \\
$P_1$ (Pa) & 101325.00 \\
$P_2$ (Pa) & 89166.46 \\
$P_3$ (Pa) & 101325.00 \\
$\Delta P$ (Pa) & 12158.54 \\
$\Delta h$ (m) & 0.098365 \\
$h_L$ (m) & 0.000000 \\
$Re$ & 127323.95 \\
\hline
\end{tabular}
\caption{Resultados da simulação: Caso 1 - Modo Ideal}
\end{table}

\subsection{Simulação: Caso 2 - Modo Realista - Água}

\subsubsection{Parâmetros de Entrada}

\begin{itemize}
    \item Modo de operação: \textbf{Realista}
    \item Diâmetro de entrada: $D_1 = 0.1$ m
    \item Diâmetro da garganta: $D_2 = 0.05$ m
    \item Comprimento da garganta: $L_{garganta} = 0.5$ m
    \item Fluido selecionado: Água (densidade $\rho$ e viscosidade $\mu$ calculadas automaticamente)
    \item Temperatura: $T = 20$ °C
    \item Pressão na entrada: $P_1 = 0$ Pa (manométricos, equivalente a pressão atmosférica)
    \item Densidade do fluido manométrico: $\rho_m = 13600$ kg/m³
    \item Variável de entrada: Vazão volumétrica $Q = 0.01$ m³/s
    \item Material do tubo: Aço comercial (fator de atrito $f$ calculado automaticamente)
\end{itemize}

\textbf{Nota:} As propriedades do fluido (densidade $\rho$ e viscosidade dinâmica $\mu$) são calculadas automaticamente pela biblioteca \texttt{thermo} com base no fluido selecionado, temperatura e pressão. O fator de atrito $f$ é calculado automaticamente no modo Realista com base no material do tubo, número de Reynolds e rugosidade.

\subsubsection{Cálculos Intermediários}

\paragraph{Áreas e Velocidades:}
\begin{align*}
    A_1 &= 0.007854 \text{ m}^2, \quad A_2 = 0.001963 \text{ m}^2 \\
    v_1 &= 1.2732 \text{ m/s}, \quad v_2 = 5.0930 \text{ m/s}
\end{align*}

\paragraph{Coeficiente de Perda na Entrada:}
$k_{entrada} = 0.04$ (modo Realista - há perdas por atrito e dissipação)

\paragraph{Pressão na Garganta - Derivação Passo a Passo (Modo Realista):}

No modo realista, há perdas na entrada que devem ser consideradas. Vamos derivar a fórmula passo a passo:

\subparagraph{Passo 1: Equação de Bernoulli com Perdas}

A equação de Bernoulli modificada, incluindo perdas na entrada, é:
\begin{equation}
P_1 + \frac{1}{2}\rho v_1^2 = P_2 + \frac{1}{2}\rho v_2^2 + \Delta P_{entrada}
\end{equation}

onde $\Delta P_{entrada}$ é a perda de pressão na entrada.

\subparagraph{Passo 2: Modelagem da Perda na Entrada}

A perda na entrada é modelada como uma fração da pressão dinâmica na garganta:
\begin{equation}
\Delta P_{entrada} = k_{entrada} \cdot \frac{1}{2}\rho v_2^2
\end{equation}

No modo realista, $k_{entrada} = 0.04$ (perda típica para entrada bem projetada).

\subparagraph{Passo 3: Substituição da Perda na Equação}

Substituindo na equação de Bernoulli:
\begin{equation}
P_1 + \frac{1}{2}\rho v_1^2 = P_2 + \frac{1}{2}\rho v_2^2 + k_{entrada} \cdot \frac{1}{2}\rho v_2^2
\end{equation}

\subparagraph{Passo 4: Fatoração}

Fatorando o termo $\frac{1}{2}\rho v_2^2$:
\begin{equation}
P_1 + \frac{1}{2}\rho v_1^2 = P_2 + \frac{1}{2}\rho v_2^2 (1 + k_{entrada})
\end{equation}

\subparagraph{Passo 5: Isolamento de $P_2$}

Subtraindo $\frac{1}{2}\rho v_2^2 (1 + k_{entrada})$ de ambos os lados:
\begin{equation}
P_2 = P_1 + \frac{1}{2}\rho v_1^2 - \frac{1}{2}\rho v_2^2 (1 + k_{entrada})
\end{equation}

\subparagraph{Passo 6: Fatoração Final}

Fatorando $\frac{1}{2}\rho$:
\begin{equation}
P_2 = P_1 - \frac{1}{2}\rho \left[v_2^2(1 + k_{entrada}) - v_1^2\right]
\end{equation}

Esta é a fórmula utilizada no código para o modo realista!

\subparagraph{Passo 7: Aplicação Numérica}

Substituindo os valores calculados:
\begin{align*}
    P_2 &= P_1 - \frac{1}{2}\rho \left[v_2^2(1 + k_{entrada}) - v_1^2\right] \\
    P_2 &= 101325 - \frac{1}{2} \cdot 1000 \left[(5.0930)^2 \cdot (1 + 0.04) - (1.2732)^2\right] \\
    P_2 &= 101325 - \frac{1}{2} \cdot 1000 \left[5.0930^2 \cdot 1.04 - 1.2732^2\right] \\
    P_2 &= 101325 - \frac{1}{2} \cdot 1000 \left[25.939 \cdot 1.04 - 1.621\right] \\
    P_2 &= 101325 - \frac{1}{2} \cdot 1000 \left[26.977 - 1.621\right] \\
    P_2 &= 101325 - \frac{1}{2} \cdot 1000 \cdot 25.356 \\
    P_2 &= 101325 - 12677.31 \\
    P_2 &= 88647.69 \text{ Pa}
\end{align*}

\textbf{Comparação com Modo Ideal:}
\begin{itemize}
    \item \textbf{Modo Ideal:} $P_2 = 89166.46$ Pa (sem perdas na entrada)
    \item \textbf{Modo Realista:} $P_2 = 88647.69$ Pa (com perdas na entrada)
    \item \textbf{Diferença:} $89166.46 - 88647.69 = 518.77$ Pa
\end{itemize}

Esta diferença corresponde exatamente à perda na entrada: $\Delta P_{entrada} = k_{entrada} \cdot \frac{1}{2}\rho v_2^2 = 0.04 \cdot \frac{1}{2} \cdot 1000 \cdot (5.0930)^2 = 518.76$ Pa.

\paragraph{Perdas na Garganta:}
\begin{align*}
    h_{f,garganta} &= f \frac{L_{garganta}}{D_2} \frac{v_2^2}{2g} \\
    h_{f,garganta} &= 0.02 \cdot \frac{0.5}{0.05} \cdot \frac{(5.0930)^2}{2 \cdot 9.81} = 0.264406 \text{ m} \\
    \Delta P_{garganta} &= h_{f,garganta} \cdot \rho \cdot g = 0.264406 \cdot 1000 \cdot 9.81 = 2593.82 \text{ Pa}
\end{align*}

\paragraph{Pressão Final na Garganta:}
\begin{align*}
    P_{2,fim} &= P_2 - \Delta P_{garganta} = 88647.69 - 2593.82 = 86053.87 \text{ Pa}
\end{align*}

\paragraph{Cálculo da Pressão na Saída ($P_3$):}
\begin{align*}
    \Delta P_{recuperacao} &= \frac{1}{2}\rho(v_2^2 - v_1^2) = \frac{1}{2} \cdot 1000 \left[(5.0930)^2 - (1.2732)^2\right] = 12158.54 \text{ Pa} \\
    K_{difusor} &= 0.2940 \\
    \Delta P_{difusor} &= K_{difusor} \cdot \frac{1}{2}\rho v_2^2 = 0.2940 \cdot \frac{1}{2} \cdot 1000 \cdot (5.0930)^2 = 3812.92 \text{ Pa} \\
    P_3 &= P_{2,fim} + \Delta P_{recuperacao} - \Delta P_{difusor} \\
    P_3 &= 86053.87 + 12158.54 - 3812.92 = 94399.49 \text{ Pa}
\end{align*}

\paragraph{Perdas Totais:}
\begin{align*}
    \Delta P_{entrada} &= k_{entrada} \cdot \frac{1}{2}\rho v_2^2 = 0.04 \cdot \frac{1}{2} \cdot 1000 \cdot (5.0930)^2 = 518.76 \text{ Pa} \\
    \Delta P_{garganta} &= 2593.82 \text{ Pa} \\
    \Delta P_{difusor} &= 3812.92 \text{ Pa} \\
    \Delta P_{total} &= 6925.50 \text{ Pa} \\
    h_L &= \frac{\Delta P_{total}}{\rho g} = \frac{6925.50}{1000 \cdot 9.81} = 0.705964 \text{ m}
\end{align*}

\paragraph{Desnível Manométrico:}
\begin{align*}
    \Delta h &= \frac{\Delta P}{(\rho_m - \rho) g} = \frac{12677.31}{(13600 - 1000) \cdot 9.81} = 0.102562 \text{ m}
\end{align*}

\subsubsection{Resumo dos Resultados}

\begin{table}[h]
\centering
\begin{tabular}{l|c}
\hline
\textbf{Grandeza} & \textbf{Valor} \\
\hline
$A_1$ (m²) & 0.007854 \\
$A_2$ (m²) & 0.001963 \\
$v_1$ (m/s) & 1.2732 \\
$v_2$ (m/s) & 5.0930 \\
$P_1$ (Pa) & 101325.00 \\
$P_2$ (Pa) & 88647.69 \\
$P_{2,fim}$ (Pa) & 86053.87 \\
$P_3$ (Pa) & 94399.49 \\
$\Delta P$ (Pa) & 12677.31 \\
$\Delta h$ (m) & 0.102562 \\
$h_L$ (m) & 0.705964 \\
$Re$ & 127323.95 \\
\hline
\end{tabular}
\caption{Resultados da simulação: Caso 2 - Modo Realista}
\end{table}

\subsection{Simulação: Caso 3 - Modo Realista - Alta Vazão}

\subsubsection{Parâmetros de Entrada}

\begin{itemize}
    \item Modo de operação: \textbf{Realista}
    \item Diâmetro de entrada: $D_1 = 0.15$ m
    \item Diâmetro da garganta: $D_2 = 0.075$ m
    \item Comprimento da garganta: $L_{garganta} = 0.5$ m
    \item Fluido selecionado: Água (densidade $\rho$ e viscosidade $\mu$ calculadas automaticamente)
    \item Temperatura: $T = 20$ °C
    \item Pressão na entrada: $P_1 = 200000$ Pa (manométricos)
    \item Densidade do fluido manométrico: $\rho_m = 13600$ kg/m³
    \item Variável de entrada: Vazão volumétrica $Q = 0.025$ m³/s (alta vazão)
    \item Material do tubo: Aço comercial (fator de atrito $f$ calculado automaticamente)
\end{itemize}

\textbf{Nota:} As propriedades do fluido (densidade $\rho$ e viscosidade dinâmica $\mu$) são calculadas automaticamente pela biblioteca \texttt{thermo} com base no fluido selecionado, temperatura e pressão. O fator de atrito $f$ é calculado automaticamente no modo Realista com base no material do tubo, número de Reynolds e rugosidade.

\subsubsection{Resultados Principais}

\begin{table}[h]
\centering
\begin{tabular}{l|c}
\hline
\textbf{Grandeza} & \textbf{Valor} \\
\hline
$A_1$ (m²) & 0.017671 \\
$A_2$ (m²) & 0.004418 \\
$v_1$ (m/s) & 1.4147 \\
$v_2$ (m/s) & 5.6588 \\
$P_1$ (Pa) & 200000.00 \\
$P_2$ (Pa) & 184349.00 \\
$P_{2,fim}$ (Pa) & 182427.65 \\
$P_3$ (Pa) & 192730.89 \\
$\Delta P$ (Pa) & 15651.00 \\
$\Delta h$ (m) & 0.126620 \\
$h_L$ (m) & 0.740989 \\
$Re$ & 212206.59 \\
\hline
\end{tabular}
\caption{Resultados da simulação: Caso 3 - Alta Vazão}
\end{table}

\textbf{Observações:}
\begin{itemize}
    \item Com maior vazão, as velocidades aumentam proporcionalmente
    \item A diferença de pressão $\Delta P$ aumenta com o quadrado da velocidade
    \item As perdas de carga ($h_L = 0.576$ m) são maiores devido às maiores velocidades
    \item O número de Reynolds aumenta, indicando escoamento mais turbulento
\end{itemize}

\subsection{Simulação: Caso 4 - Modo Realista - Razão de Áreas Pequena}

\subsubsection{Parâmetros de Entrada}

\begin{itemize}
    \item Modo de operação: \textbf{Realista}
    \item Diâmetro de entrada: $D_1 = 0.12$ m
    \item Diâmetro da garganta: $D_2 = 0.10$ m (razão $\beta = 0.833$)
    \item Comprimento da garganta: $L_{garganta} = 0.5$ m
    \item Fluido selecionado: Água (densidade $\rho$ e viscosidade $\mu$ calculadas automaticamente)
    \item Temperatura: $T = 20$ °C
    \item Pressão na entrada: $P_1 = 0$ Pa (manométricos, equivalente a pressão atmosférica)
    \item Densidade do fluido manométrico: $\rho_m = 13600$ kg/m³
    \item Variável de entrada: Vazão volumétrica $Q = 0.015$ m³/s
    \item Material do tubo: Aço comercial (fator de atrito $f$ calculado automaticamente)
\end{itemize}

\textbf{Nota:} As propriedades do fluido (densidade $\rho$ e viscosidade dinâmica $\mu$) são calculadas automaticamente pela biblioteca \texttt{thermo} com base no fluido selecionado, temperatura e pressão. O fator de atrito $f$ é calculado automaticamente no modo Realista com base no material do tubo, número de Reynolds e rugosidade.

\subsubsection{Resultados Principais}

\begin{table}[h]
\centering
\begin{tabular}{l|c}
\hline
\textbf{Grandeza} & \textbf{Valor} \\
\hline
$A_1$ (m²) & 0.011310 \\
$A_2$ (m²) & 0.007854 \\
$v_1$ (m/s) & 1.3263 \\
$v_2$ (m/s) & 1.9099 \\
$P_1$ (Pa) & 101325.00 \\
$P_2$ (Pa) & 100307.79 \\
$P_{2,fim}$ (Pa) & 100125.41 \\
$P_3$ (Pa) & 100843.48 \\
$\Delta P$ (Pa) & 1017.21 \\
$\Delta h$ (m) & 0.008229 \\
$h_L$ (m) & 0.049084 \\
$Re$ & 159154.94 \\
\hline
\end{tabular}
\caption{Resultados da simulação: Caso 4 - Razão de Áreas Pequena}
\end{table}

\textbf{Observações:}
\begin{itemize}
    \item Com menor diferença entre diâmetros, a diferença de pressão é menor ($\Delta P = 1017$ Pa vs $12158$ Pa no Caso 1)
    \item As perdas são menores devido às menores velocidades
    \item O coeficiente de perda do difusor ($K_{difusor} = 0.1240$) é menor que no caso com maior razão de áreas
    \item A recuperação de pressão é quase total ($P_3 \approx P_1$)
\end{itemize}

\section{Análise Comparativa}

\subsection{Comparação entre Modos Ideal e Realista}

A Tabela \ref{tab:comparacao} apresenta uma comparação entre os resultados obtidos nos modos ideal e realista para as mesmas condições de entrada (Casos 1 e 2).

\begin{table}[h]
\centering
\begin{tabular}{|l|c|c|}
\hline
\textbf{Grandeza} & \textbf{Modo Ideal} & \textbf{Modo Realista} \\
\hline
$P_2$ (Pa) & 89166.46 & 88647.69 \\
$P_3$ (Pa) & 101325.00 (recuperação total) & 94399.49 (perda de 6925 Pa) \\
$h_L$ (m) & 0.000000 & 0.705964 \\
$\Delta P$ (Pa) & 12158.54 & 12677.31 \\
$\Delta h$ (m) & 0.098365 & 0.102562 \\
Perdas & Nenhuma & Entrada + Garganta + Difusor \\
\hline
\end{tabular}
\caption{Comparação entre modos de operação}
\label{tab:comparacao}
\end{table}

\textbf{Principais diferenças:}
\begin{itemize}
    \item No modo ideal, $P_3 = P_1$ (recuperação total de pressão)
    \item No modo realista, há uma perda permanente de $P_1 - P_3 = 6925$ Pa
    \item A perda de carga total no modo realista é $h_L = 0.706$ m
    \item A diferença de pressão é ligeiramente maior no modo realista devido às perdas na entrada
\end{itemize}

\subsection{Influência dos Parâmetros}

\subsubsection{Influência da Razão de Diâmetros}

A razão $\beta = D_2/D_1$ afeta significativamente:
\begin{itemize}
    \item \textbf{Diferença de pressão:} Maior $\beta$ (menor diferença entre diâmetros) resulta em menor $\Delta P$
    \item \textbf{Perdas no difusor:} Maior razão de áreas ($AR = 1/\beta^2$) resulta em maiores perdas
    \item \textbf{Recuperação de pressão:} Com menor $\beta$, a recuperação é mais eficiente
\end{itemize}

\subsubsection{Influência da Vazão}

A vazão $Q$ influencia:
\begin{itemize}
    \item \textbf{Velocidades:} Proporcionais à vazão ($v = Q/A$)
    \item \textbf{Diferença de pressão:} Proporcional ao quadrado da vazão ($\Delta P \propto v^2$)
    \item \textbf{Perdas de carga:} Proporcionais ao quadrado da velocidade, portanto ao quadrado da vazão
    \item \textbf{Número de Reynolds:} Proporcional à vazão
\end{itemize}

\subsubsection{Influência do Fator de Atrito}

\textbf{Nota:} O fator de atrito $f$ é utilizado apenas no modo realista. No modo ideal, não há atrito e o fator de atrito não é utilizado.

No modo realista, o fator de atrito $f$ afeta diretamente:
\begin{itemize}
    \item \textbf{Perdas na garganta:} $h_f = f \frac{L}{D} \frac{v^2}{2g}$
    \item \textbf{Pressão final na garganta:} $P_{2,fim} = P_2 - \Delta P_{garganta}$
    \item \textbf{Perda de carga total:} Contribui diretamente para $h_L$
\end{itemize}

\subsubsection{Influência do Coeficiente de Descarga}

O coeficiente de descarga $C_d$ (usado no modo realista):
\begin{itemize}
    \item Representa a eficiência do medidor
    \item Valores típicos: $0.95 < C_d < 0.99$
    \item No código atual, não é utilizado diretamente nos cálculos de pressão, mas poderia ser usado para corrigir a vazão teórica
\end{itemize}

\section{Resumo: Mapeamento dos Conteúdos Teóricos no Código}

Esta seção apresenta um resumo completo de onde cada conteúdo teórico da mecânica dos fluidos é aplicado no código do simulador de Venturi.

\subsection{Definição de Fluido, Formas de Análise e Descrição}

\begin{itemize}
    \item \textbf{Fluido incompressível:} A densidade $\rho$ é constante em todos os cálculos (parâmetro de entrada, linha 9).
    \item \textbf{Abordagem de Euler:} Propriedades calculadas em seções fixas (entrada, garganta, saída) - linhas 31-32, 39, 47, 52, 58.
    \item \textbf{Escoamento permanente:} Cálculos realizados uma única vez, assumindo regime permanente (método \texttt{calcular()}, linha 9).
    \item \textbf{Modo invíscido vs viscoso:} Implementado através do parâmetro \texttt{mode} (linhas 34-37, 41-42, 51-53).
\end{itemize}

\subsection{Propriedades dos Fluidos}

\begin{itemize}
    \item \textbf{Densidade ($\rho$):} Utilizada em:
    \begin{itemize}
        \item Cálculo de pressões (linha 39, 52, 55, 57, 58, 59)
        \item Cálculo de perdas (linhas 45, 61)
        \item Cálculo do desnível manométrico (linha 63)
        \item Cálculo do número de Reynolds (linha 71)
    \end{itemize}
    \item \textbf{Viscosidade dinâmica ($\mu$):} Utilizada no cálculo do número de Reynolds (linha 71).
    \item \textbf{Pressão ($P$):} Calculada em múltiplas seções:
    \begin{itemize}
        \item $P_1$: Parâmetro de entrada (linha 9)
        \item $P_2$: Calculada na linha 39
        \item $P_{2,fim}$: Calculada na linha 47
        \item $P_3$: Calculada na linha 52 (ideal) ou 58 (realista)
    \end{itemize}
\end{itemize}

\subsection{Forças Hidrostáticas e Empuxo}

\begin{itemize}
    \item \textbf{Princípio hidrostático:} Aplicado no cálculo do desnível manométrico (linha 63):
    \begin{equation}
    \Delta h = \frac{\Delta P}{(\rho_m - \rho) g}
    \end{equation}
    \item \textbf{Diferença de densidades:} A diferença $(\rho_m - \rho)$ é fundamental para o funcionamento do manômetro diferencial.
\end{itemize}

\subsection{Equações Básicas na Forma Integral para um Volume de Controle}

\begin{itemize}
    \item \textbf{Conservação de Massa (Equação da Continuidade):} Aplicada diretamente nas linhas 31-32:
    \begin{lstlisting}[firstnumber=31]
self.v1 = self.Q / self.A1
self.v2 = self.Q / self.A2
    \end{lstlisting}
    Garante que $v_1 A_1 = v_2 A_2 = Q$ (constante).
    \item \textbf{Conservação de Quantidade de Movimento:} Implícita no cálculo das pressões e forças nas paredes do Venturi.
    \item \textbf{Conservação de Energia:} Base teórica para a equação de Bernoulli aplicada no código.
\end{itemize}

\subsection{Equação de Euler}

\begin{itemize}
    \item \textbf{Integração ao longo de linha de corrente:} Resulta na equação de Bernoulli, que é a base para o cálculo de pressões.
    \item \textbf{Aplicação:} A equação de Euler integrada é utilizada indiretamente através da equação de Bernoulli (linha 39).
\end{itemize}

\subsection{Equação de Bernoulli}

\begin{itemize}
    \item \textbf{Forma simplificada (tubo horizontal):} Base para o cálculo de $P_2$ (linha 39):
    \begin{equation}
    P_2 = P_1 - \frac{1}{2}\rho \left[v_2^2(1 + k_{entrada}) - v_1^2\right]
    \end{equation}
    \item \textbf{Equação modificada (com perdas):} Aplicada em múltiplos pontos:
    \begin{itemize}
        \item Cálculo de $P_2$ com perda na entrada (linha 39)
        \item Recuperação dinâmica (linha 55): $\Delta P_{recuperacao} = \frac{1}{2}\rho(v_2^2 - v_1^2)$
        \item Perda no difusor (linha 57): $\Delta P_{difusor} = K_{difusor} \cdot \frac{1}{2}\rho v_2^2$
        \item Cálculo de $P_3$ (linha 58): $P_3 = P_{2,fim} + \Delta P_{recuperacao} - \Delta P_{difusor}$
    \end{itemize}
\end{itemize}

\subsection{Escoamento em Dutos}

\begin{itemize}
    \item \textbf{Perdas por atrito (Darcy-Weisbach):} Implementada no método \texttt{\_calcular\_perda\_carga\_garganta()} (linhas 65-68):
    \begin{equation}
    h_{f,garganta} = f \frac{L_{garganta}}{D_2} \frac{v_2^2}{2g}
    \end{equation}
    \item \textbf{Perdas localizadas:}
    \begin{itemize}
        \item Perda na entrada: $k_{entrada} = 0.04$ (modo realista, linha 37), aplicada na linha 39 e calculada na linha 59
        \item Perda no difusor: $K_{difusor}$ calculado no método \texttt{\_obter\_k\_difusor\_15\_graus()} (linhas 98-115), aplicado na linha 57
    \end{itemize}
    \item \textbf{Número de Reynolds:} Calculado no método \texttt{calcular\_reynolds()} (linha 71):
    \begin{equation}
    Re = \frac{\rho v_1 D_1}{\mu}
    \end{equation}
    \item \textbf{Velocidade média:} Utilizada em todos os cálculos (linhas 31-32), assumindo perfil de velocidade uniforme (aproximação válida para escoamento turbulento).
\end{itemize}

\subsection{Tabela Resumo de Aplicações}

\begin{table}[h]
\centering
\small
\begin{tabular}{|p{4cm}|p{10cm}|}
\hline
\textbf{Conteúdo Teórico} & \textbf{Aplicação no Código} \\
\hline
Fluido incompressível & Densidade $\rho$ constante (parâmetro linha 9) \\
\hline
Abordagem de Euler & Propriedades em seções fixas (linhas 31-32, 39, 47, 52, 58) \\
\hline
Densidade ($\rho$) & Cálculo de pressões, perdas, desnível, Reynolds (linhas 39, 45, 55, 57, 59, 61, 63, 71) \\
\hline
Viscosidade ($\mu$) & Cálculo do número de Reynolds (linha 71) \\
\hline
Pressão ($P$) & $P_1$ entrada (linha 9), $P_2$ calculada (linha 39), $P_3$ calculada (linhas 52, 58) \\
\hline
Princípio hidrostático & Cálculo do desnível manométrico (linha 63) \\
\hline
Conservação de massa & Equação da continuidade (linhas 31-32): $v = Q/A$ \\
\hline
Equação de Euler & Base teórica para Bernoulli (aplicada indiretamente) \\
\hline
Equação de Bernoulli & Cálculo de $P_2$ (linha 39), recuperação dinâmica (linha 55), perdas (linhas 57, 59) \\
\hline
Darcy-Weisbach & Perdas na garganta (linhas 65-68) \\
\hline
Perdas localizadas & Entrada: $k_{entrada}$ (linhas 37, 39, 59); Difusor: $K_{difusor}$ (linhas 98-115, 57) \\
\hline
Número de Reynolds & Caracterização do regime (linha 71) \\
\hline
Velocidade média & Utilizada em todos os cálculos (linhas 31-32) \\
\hline
\end{tabular}
\caption{Resumo do mapeamento dos conteúdos teóricos no código}
\label{tab:mapeamento}
\end{table}

\section{Conclusão}

Este documento apresentou uma explicação detalhada de todos os cálculos implementados no simulador de Venturi. Os métodos implementados seguem os princípios fundamentais da mecânica dos fluidos:

\begin{itemize}
    \item \textbf{Conservação de Massa:} Utilizada no cálculo das velocidades através da equação da continuidade
    \item \textbf{Equação de Bernoulli:} Base para o cálculo das pressões ao longo do medidor
    \item \textbf{Perdas de Carga:} Consideradas através de correlações empíricas e a equação de Darcy-Weisbach
    \item \textbf{Geometria:} Cálculo automático das dimensões baseado em recomendações de projeto
\end{itemize}

O simulador permite analisar tanto o comportamento ideal (sem atrito, sem perdas) quanto o real (com atrito e perdas de carga) do medidor Venturi, fornecendo informações detalhadas sobre pressões, velocidades e perdas de carga em cada seção do dispositivo.

\end{document}

