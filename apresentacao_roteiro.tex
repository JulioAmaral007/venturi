\documentclass[12pt,a4paper]{article}
\usepackage[utf8]{inputenc}
\usepackage[portuguese]{babel}
\usepackage{geometry}
\usepackage{enumitem}
\usepackage{hyperref}
\usepackage{xcolor}
\usepackage{titlesec}

\geometry{margin=2.5cm}
\setlength{\parskip}{0.5em}

% Cores para destaque
\definecolor{actioncolor}{RGB}{0, 100, 0}
\definecolor{scriptcolor}{RGB}{0, 0, 120}

\title{\textbf{Roteiro de Apresentação: Simulador de Medidor de Venturi}}
\author{Defesa de Trabalho - Fenômenos de Transporte}
\date{\today}

\begin{document}

\maketitle

\section*{Estrutura da Apresentação}
\textbf{Tempo Estimado:} 10 a 15 minutos \\
\textbf{Objetivo:} Demonstrar a aplicação prática dos conceitos de Mecânica dos Fluidos (Hidrostática, Bernoulli, Continuidade e Perda de Carga) através de um software de simulação.

\hrulefill

\section{Introdução (2 minutos)}

\subsection*{O que Falar (Script Sugerido)}
\textcolor{scriptcolor}{"Bom dia/Boa noite a todos.
O objetivo deste trabalho foi desenvolver uma aplicação prática que conectasse a teoria de sala de aula com a realidade da engenharia.
Para isso, desenvolvi um \textbf{Simulador de Medidor de Venturi}.
Escolhi este dispositivo porque ele integra, em um único sistema, quase todos os conceitos que estudamos no semestre: desde a estática dos fluidos no manômetro até a dinâmica de escoamento e perda de carga nos dutos.
O software não apenas calcula os valores, mas permite visualizar os fenômenos físicos acontecendo em tempo real."}

\subsection*{Ação no Software}
\begin{itemize}
\begin{itemize}
    \item \textcolor{actioncolor}{\textbf{Ação:} Selecionar Modo 'Simulação Interativa' $\to$ Tipo 'Ideal'.}
    \item \textcolor{actioncolor}{\textbf{Ação:} Aumentar a Vazão ($Q$).}
\end{itemize}
\textcolor{scriptcolor}{"Vejam no gráfico de Pressão: conforme o fluido acelera na garganta, a pressão cai. Isso é a visualização direta da conversão de energia prevista por Bernoulli."}

\subsection*{Cenário 2: Aplicação Industrial (Modo Medidor)}
\begin{itemize}
\documentclass[12pt,a4paper]{article}
\usepackage[utf8]{inputenc}
\usepackage[portuguese]{babel}
\usepackage{geometry}
\usepackage{enumitem}
\usepackage{hyperref}
\usepackage{xcolor}
\usepackage{titlesec}

\geometry{margin=2.5cm}
\setlength{\parskip}{0.5em}

% Cores para destaque
\definecolor{actioncolor}{RGB}{0, 100, 0}
\definecolor{scriptcolor}{RGB}{0, 0, 120}

\title{\textbf{Roteiro de Apresentação: Simulador de Medidor de Venturi}}
\author{Defesa de Trabalho - Fenômenos de Transporte}
\date{\today}

\begin{document}

\maketitle

\section*{Estrutura da Apresentação}
\textbf{Tempo Estimado:} 10 a 15 minutos \\
\textbf{Objetivo:} Demonstrar a aplicação prática dos conceitos de Mecânica dos Fluidos (Hidrostática, Bernoulli, Continuidade e Perda de Carga) através de um software de simulação.

\hrulefill

\section{Introdução (2 minutos)}

\subsection*{O que Falar (Script Sugerido)}
\textcolor{scriptcolor}{"Bom dia/Boa noite a todos.
O objetivo deste trabalho foi desenvolver uma aplicação prática que conectasse a teoria de sala de aula com a realidade da engenharia.
Para isso, desenvolvi um \textbf{Simulador de Medidor de Venturi}.
Escolhi este dispositivo porque ele integra, em um único sistema, quase todos os conceitos que estudamos no semestre: desde a estática dos fluidos no manômetro até a dinâmica de escoamento e perda de carga nos dutos.
O software não apenas calcula os valores, mas permite visualizar os fenômenos físicos acontecendo em tempo real."}

\subsection*{Ação no Software}
\begin{itemize}
\begin{itemize}
    \item \textcolor{actioncolor}{\textbf{Ação:} Selecionar Modo 'Simulação Interativa' $\to$ Tipo 'Ideal'.}
    \item \textcolor{actioncolor}{\textbf{Ação:} Aumentar a Vazão ($Q$).}
\end{itemize}
\textcolor{scriptcolor}{"Vejam no gráfico de Pressão: conforme o fluido acelera na garganta, a pressão cai. Isso é a visualização direta da conversão de energia prevista por Bernoulli."}

\subsection*{Cenário 2: Aplicação Industrial (Modo Medidor)}
\begin{itemize}
    \item \textcolor{actioncolor}{\textbf{Ação:} Mudar para Tipo 'Medidor'.}
    \item \textcolor{actioncolor}{\textbf{Ação:} Ajustar o slider de Desnível ($\Delta h$).}
\end{itemize}
\textcolor{scriptcolor}{"Este é o modo mais importante. Na indústria, não sabemos a vazão; nós lemos o manômetro.
Aqui, eu altero o desnível do mercúrio (Hidrostática) e o software me diz qual é a vazão no tubo. Isso prova a aplicação prática dos conceitos de empuxo e manometria."}

\subsection*{Cenário 3: Realidade e Reynolds (Modo Realista)}
\begin{itemize}
    \item \textcolor{actioncolor}{\textbf{Ação:} Mudar para Tipo 'Realista'.}
    \item \textcolor{actioncolor}{\textbf{Ação:} Mostrar a aba 'Dados Completos' ou 'Energia'.}
\end{itemize}
\textcolor{scriptcolor}{"No modo realista, a linha de energia (roxa) não é constante, ela decai. Essa inclinação representa a perda de carga ($h_L$) que calculamos no código.
Além disso, o sistema calcula o \textbf{Número de Reynolds} em tempo real, nos alertando se o regime é Laminar ou Turbulento, o que é vital para a precisão da medição."}

\subsection*{Passo a Passo da Dedução (Mostrar Slide ou Quadro)}

\subsubsection{1. O Ponto de Partida: Continuidade}
\textcolor{scriptcolor}{"Tudo começa com a \textbf{Equação da Continuidade}. Para um fluido incompressível:
\[ Q = A_1 v_1 = A_2 v_2 \]
Onde:
\begin{itemize}
    \item $Q$: Vazão volumétrica ($m^3/s$)
    \item $A_1, A_2$: Áreas das seções de entrada e da garganta ($m^2$)
    \item $v_1, v_2$: Velocidades médias do fluido ($m/s$)
\end{itemize}
Isso prova que na garganta ($A_2 < A_1$), a velocidade deve aumentar ($v_2 > v_1$)."
}

\subsubsection{2. A Conservação de Energia: Bernoulli}
\textcolor{scriptcolor}{"A segunda lei é a \textbf{Equação de Bernoulli}. Para um tubo horizontal ($z_1=z_2$):
\[ \frac{P_1}{\rho} + \frac{v_1^2}{2} = \frac{P_2}{\rho} + \frac{v_2^2}{2} \]
Onde:
\begin{itemize}
    \item $P_1, P_2$: Pressões estáticas ($Pa$)
    \item $\rho$: Densidade do fluido ($kg/m^3$)
\end{itemize}
Rearranjando, vemos que a diferença de pressão impulsiona o aumento de velocidade:
\[ P_1 - P_2 = \frac{\rho}{2} (v_2^2 - v_1^2) \]
"}

\subsubsection{3. A Substituição}
\textcolor{scriptcolor}{"Substituindo $v_1$ da continuidade na equação de energia, isolamos a velocidade na garganta:
\[ v_2 = \sqrt{ \frac{2(P_1 - P_2)}{\rho (1 - \beta^4)} } \]
Onde $\beta = D_2/D_1$ é a razão entre os diâmetros do medidor."}

\subsubsection{4. A Equação Final do Medidor}
\textcolor{scriptcolor}{"Finalmente, chegamos à equação programada no simulador. Substituímos $\Delta P$ pela leitura do manômetro:
\[ \Delta P = (\rho_m - \rho) g \Delta h \]
Onde:
\begin{itemize}
    \item $\rho_m$: Densidade do fluido manométrico (ex: mercúrio)
    \item $g$: Aceleração da gravidade ($9.81 m/s^2$)
    \item $\Delta h$: Desnível medido ($m$)
\end{itemize}
E adicionamos o coeficiente de descarga $C_d$ para corrigir perdas reais, chegando à fórmula final:
\[ Q = C_d A_2 \sqrt{ \frac{2 g \Delta h (\rho_m - \rho)}{\rho (1 - \beta^4)} } \]
É esta equação exata que o código processa."}

\hrulefill

\section{Conclusão (2 minutos)}

\subsection*{O que Falar (Script Sugerido)}
\textcolor{scriptcolor}{"Para concluir, este trabalho serviu para consolidar todo o conteúdo da disciplina.
Foi necessário entender profundamente as propriedades dos fluidos, a estática para o manômetro, e a dinâmica para o escoamento, a fim de programar corretamente a física do problema.
O resultado é uma ferramenta que pode ser usada para validar qualquer exercício de sala de aula.
Obrigado."}

\end{document}
