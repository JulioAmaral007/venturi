\documentclass{article}
\usepackage[utf8]{inputenc}
\usepackage{amsmath}
\usepackage[brazil]{babel}
\usepackage{geometry}
\geometry{a4paper, margin=2.5cm}

\title{Formulário e Deduções -- Simulador de Venturi}
\date{}

\begin{document}

\maketitle

\section{Dedução da Equação do Medidor de Venturi}

A equação principal utilizada no simulador para calcular a vazão ($Q$) a partir da diferença de pressão ($\Delta P$) ou desnível ($\Delta h$) é derivada dos princípios de conservação de massa e energia.

\subsection{1. Equações Governantes}
Para um fluido incompressível em regime permanente escoando por um tubo horizontal, aplicamos:

\textbf{Equação da Continuidade (Conservação de Massa):}
\[
Q = A_1 v_1 = A_2 v_2 \implies v_1 = v_2 \frac{A_2}{A_1}
\]

\textbf{Equação de Bernoulli (Conservação de Energia - Ideal):}
Considerando $z_1 = z_2$ (tubo horizontal) e sem perdas ($h_L=0$):
\[
\frac{P_1}{\rho} + \frac{v_1^2}{2} = \frac{P_2}{\rho} + \frac{v_2^2}{2}
\]
Rearranjando para a diferença de pressão:
\[
\frac{P_1 - P_2}{\rho} = \frac{v_2^2 - v_1^2}{2}
\]

\subsection{2. Substituição e Solução}
Substituindo $v_1$ pela expressão da continuidade ($v_1 = v_2 \frac{A_2}{A_1}$):
\[
\frac{P_1 - P_2}{\rho} = \frac{v_2^2 - \left(v_2 \frac{A_2}{A_1}\right)^2}{2}
\]
\[
\frac{2(P_1 - P_2)}{\rho} = v_2^2 \left[ 1 - \left(\frac{A_2}{A_1}\right)^2 \right]
\]
Isolando $v_2$:
\[
v_2 = \sqrt{ \frac{2(P_1 - P_2)}{\rho \left[ 1 - \left(\frac{A_2}{A_1}\right)^2 \right]} }
\]

\subsection{3. Vazão Teórica ($Q_{\text{teórica}}$)}
Como $Q = A_2 v_2$, temos:
\[
Q_{\text{teórica}} = A_2 \sqrt{ \frac{2 \Delta P}{\rho (1 - \beta^4)} }
\]
onde $\Delta P = P_1 - P_2$ e $\beta = D_2/D_1 = \sqrt{A_2/A_1}$.

\subsection{4. Vazão Real ($Q_{\text{real}}$)}
Para considerar as perdas de energia reais (atrito, turbulência), introduzimos o coeficiente de descarga $C_d$:
\[
Q_{\text{real}} = C_d \cdot Q_{\text{teórica}}
\]
\[
\boxed{Q = C_d A_2 \sqrt{ \frac{2 \Delta P}{\rho \left(1 - \left(\frac{A_2}{A_1}\right)^2\right)} }}
\]

\subsection{5. Relação com o Manômetro ($\Delta h$)}
A diferença de pressão é medida por um manômetro em U. Pela hidrostática:
\[
P_1 + \rho g z = P_2 + \rho g (z - \Delta h) + \rho_m g \Delta h
\]
Simplificando para o Venturi horizontal:
\[
P_1 - P_2 = (\rho_m - \rho) g \Delta h
\]
Substituindo $\Delta P$ na equação da vazão, chegamos à fórmula final implementada no código (Modo Medidor):
\[
\boxed{Q = C_d A_2 \sqrt{ \frac{2 g \Delta h (\rho_m - \rho)}{\rho \left(1 - \left(\frac{A_2}{A_1}\right)^2\right)} }}
\]

\section{Outras Fórmulas Implementadas}

\subsection{Perda de Carga (Modo Realista)}
No modo realista, calculamos a perda de carga distribuída $h_L$ usando Darcy-Weisbach:
\[
h_L = f \frac{L}{D_{\text{médio}}} \frac{v_{\text{médio}}^2}{2g}
\]
Isso altera a equação de Bernoulli para:
\[
\frac{P_1}{\rho g} + \frac{v_1^2}{2g} = \frac{P_2}{\rho g} + \frac{v_2^2}{2g} + h_L
\]
O simulador usa isso para calcular $\Delta P$ quando a vazão $Q$ é conhecida.

\subsection{Número de Reynolds}
\[
Re = \frac{v_1 D_1}{\nu}
\]

\end{document}
