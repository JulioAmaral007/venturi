\documentclass[12pt,a4paper]{article}

% Pacotes essenciais
\usepackage[utf8]{inputenc}
\usepackage[portuguese]{babel}
\usepackage[T1]{fontenc}
\usepackage{geometry}
\usepackage{amsmath}
\usepackage{amsfonts}
\usepackage{amssymb}
\usepackage{graphicx}
\usepackage{float}
\usepackage{booktabs}
\usepackage{siunitx}
\usepackage{hyperref}
\usepackage{indentfirst}
\usepackage{setspace}

% Configurações de margens e espaçamento
\geometry{a4paper,left=3cm,top=3cm,right=2cm,bottom=2cm}
\onehalfspacing

% Configurações de hiperlinks
\hypersetup{colorlinks=true,linkcolor=blue,citecolor=blue}

% ============================================
% INFORMAÇÕES DO DOCUMENTO
% ============================================
\title{Simulador Interativo de Medidor de Venturi:\\ Uma Ferramenta Computacional para Ensino de Mecânica dos Fluidos}
\author{Nome do Autor\\ Instituição de Ensino}
\date{\today}

\begin{document}

% ============================================
% CAPA
% ============================================
\maketitle

\thispagestyle{empty}                      % Remove numeração da capa

\begin{center}
    \vspace{2cm}
    
    \Large
    \textbf{Disciplina:} Mecânica dos Fluidos\\
    \textbf{Professor:} Nome do Professor\\
    \textbf{Data:} \today
    
    \vspace{3cm}
\end{center}

\newpage

% ============================================
% RESUMO
% ============================================
\begin{abstract}
    \noindent Este trabalho apresenta o desenvolvimento de uma aplicação web interativa para simulação de medidores de Venturi, uma ferramenta computacional desenvolvida com Python e Streamlit destinada ao ensino e compreensão de conceitos fundamentais de mecânica dos fluidos. A aplicação implementa três modos de operação distintos: modo ideal, modo realista e modo medidor, permitindo a análise de escoamentos com e sem perdas de energia. O sistema realiza cálculos baseados na equação de Bernoulli, equação da continuidade e equação de Darcy-Weisbach, considerando efeitos reais como perdas por atrito e coeficiente de descarga. A ferramenta oferece visualizações gráficas avançadas incluindo diagramas esquemáticos do medidor, representações de manômetros diferenciais, perfis de pressão ao longo do tubo e linhas de energia e piezométrica. Os resultados demonstram que a aplicação é capaz de calcular vazões com precisão adequada, analisar a influência de parâmetros geométricos e físicos no desempenho do medidor, e gerar curvas de calibração para diferentes configurações. A análise de sensibilidade ao coeficiente de descarga revelou que variações de 10\% neste parâmetro podem resultar em erros significativos na medição de vazão, destacando a importância da calibração adequada do instrumento. A aplicação desenvolvida demonstra ser uma ferramenta educacional eficaz para o ensino de conceitos de medição de vazão e análise de escoamentos em condutos, proporcionando uma compreensão visual e interativa dos fenômenos físicos envolvidos.
    
    \vspace{0.5cm}
    \noindent \textbf{Palavras-chave:} Medidor de Venturi, Mecânica dos Fluidos, Simulação Computacional, Equação de Bernoulli, Ensino de Engenharia.
\end{abstract}

\newpage

% ============================================
% SUMÁRIO
% ============================================
\tableofcontents
\newpage

% ============================================
% INTRODUÇÃO
% ============================================
\section{Introdução}

A medição de vazão em sistemas de escoamento de fluidos é uma necessidade fundamental em diversas aplicações industriais e científicas, desde o tratamento e distribuição de água até processos químicos e petroquímicos. Entre os diversos dispositivos disponíveis para esta finalidade, o medidor de Venturi destaca-se por sua precisão, baixa perda de carga permanente e robustez operacional. Desenvolvido baseado no princípio de Bernoulli, este instrumento utiliza a diferença de pressão entre duas seções de um tubo com geometria variável para determinar a vazão volumétrica do fluido em escoamento.

O ensino de conceitos relacionados a medidores de vazão apresenta desafios significativos, uma vez que os fenômenos físicos envolvidos são abstratos e requerem compreensão de princípios fundamentais de mecânica dos fluidos, incluindo conservação de massa, conservação de energia e comportamento de fluidos em regime permanente. A visualização dos perfis de pressão, linhas de energia e comportamento do escoamento ao longo do medidor são elementos cruciais para a compreensão completa do funcionamento do dispositivo.

Este trabalho apresenta o desenvolvimento de uma aplicação web interativa desenvolvida utilizando a linguagem de programação Python e o framework Streamlit, destinada a simular o comportamento de medidores de Venturi sob diferentes condições de operação. A ferramenta permite aos usuários configurar parâmetros geométricos do medidor, propriedades dos fluidos envolvidos e condições de escoamento, obtendo resultados numéricos e visualizações gráficas em tempo real. A aplicação implementa três modos distintos de simulação: o modo ideal, que considera escoamento sem perdas de energia; o modo realista, que incorpora perdas por atrito através da equação de Darcy-Weisbach; e o modo medidor, que calcula a vazão a partir de uma medição de desnível manométrico, simulando o uso prático do dispositivo.

O objetivo principal deste trabalho é desenvolver uma ferramenta educacional que facilite o ensino e a compreensão de conceitos fundamentais de mecânica dos fluidos relacionados a medidores de vazão, proporcionando uma experiência interativa que permita aos estudantes visualizar e analisar os fenômenos físicos envolvidos. Além disso, a aplicação serve como instrumento de análise técnica, permitindo a avaliação de diferentes configurações geométricas e condições operacionais, auxiliando no projeto e na calibração de medidores de Venturi para aplicações práticas.

A importância prática desta ferramenta estende-se além do ambiente acadêmico, pois medidores de Venturi são amplamente utilizados em indústrias de tratamento de água, petróleo e gás, química e geração de energia, onde a medição precisa de vazão é essencial para o controle de processos e a eficiência operacional. A capacidade de simular diferentes cenários e analisar a influência de diversos parâmetros no desempenho do medidor torna esta aplicação uma ferramenta valiosa tanto para fins educacionais quanto para aplicações práticas de engenharia.

% ============================================
% DESENVOLVIMENTO
% ============================================
\section{Desenvolvimento}

\subsection{Fundamentação Teórica}

O medidor de Venturi baseia-se no princípio de conservação de energia em escoamentos de fluidos, expresso pela equação de Bernoulli, combinado com o princípio de conservação de massa, expresso pela equação da continuidade. A geometria característica do medidor consiste em uma seção convergente, onde o diâmetro do tubo diminui gradualmente, seguida por uma garganta de diâmetro constante mínimo, e finalmente uma seção divergente onde o diâmetro retorna ao valor original.

A equação da continuidade estabelece que, para um fluido incompressível em regime permanente, a vazão volumétrica permanece constante ao longo do tubo. Esta relação pode ser expressa como:

\begin{equation}
Q = A_1 v_1 = A_2 v_2 = \text{constante}
\label{eq:continuidade}
\end{equation}

onde $Q$ representa a vazão volumétrica em metros cúbicos por segundo (\si{\meter\cubed\per\second}), $A_1$ e $A_2$ são as áreas das seções transversais nos pontos de medição em metros quadrados (\si{\meter\squared}), e $v_1$ e $v_2$ são as velocidades médias do fluido nestas seções em metros por segundo (\si{\meter\per\second}).

A equação de Bernoulli, aplicada entre dois pontos ao longo de uma linha de corrente em um escoamento incompressível, sem atrito e em regime permanente, estabelece a conservação da energia mecânica por unidade de peso do fluido:

\begin{equation}
\frac{P_1}{\rho g} + \frac{v_1^2}{2g} + z_1 = \frac{P_2}{\rho g} + \frac{v_2^2}{2g} + z_2
\label{eq:bernoulli_geral}
\end{equation}

onde $P_1$ e $P_2$ são as pressões estáticas nos pontos de medição em pascals (\si{\pascal}), $\rho$ é a densidade do fluido em quilogramas por metro cúbico (\si{\kilogram\per\meter\cubed}), $g$ é a aceleração da gravidade em metros por segundo ao quadrado (\si{\meter\per\second\squared}), e $z_1$ e $z_2$ são as alturas dos pontos em relação a um plano de referência em metros (\si{\meter}).

Para um medidor de Venturi instalado horizontalmente, as alturas $z_1$ e $z_2$ são iguais, simplificando a equação de Bernoulli para:

\begin{equation}
\frac{P_1}{\rho g} + \frac{v_1^2}{2g} = \frac{P_2}{\rho g} + \frac{v_2^2}{2g}
\label{eq:bernoulli_horizontal}
\end{equation}

Combinando as equações de continuidade e Bernoulli, é possível expressar a queda de pressão entre as duas seções em termos das velocidades ou da vazão. Rearranjando a equação de Bernoulli e substituindo a relação entre velocidades obtida da equação da continuidade, chega-se à expressão:

\begin{equation}
\Delta P = P_1 - P_2 = \frac{1}{2}\rho(v_2^2 - v_1^2)
\label{eq:delta_p_ideal}
\end{equation}

Substituindo as velocidades pela relação $v = Q/A$ obtida da equação da continuidade, a queda de pressão pode ser expressa em função da vazão:

\begin{equation}
\Delta P = \frac{1}{2}\rho Q^2\left(\frac{1}{A_2^2} - \frac{1}{A_1^2}\right)
\label{eq:delta_p_vazao}
\end{equation}

No entanto, esta equação representa o comportamento ideal do medidor, onde não há perdas de energia. Em condições reais, o escoamento apresenta perdas por atrito e outros efeitos que reduzem a eficiência do medidor. Para considerar estes efeitos, introduz-se o coeficiente de descarga $C_d$, que é a razão entre a vazão real e a vazão teórica ideal:

\begin{equation}
C_d = \frac{Q_{\text{real}}}{Q_{\text{teórico}}}
\label{eq:coeficiente_descarga}
\end{equation}

O coeficiente de descarga é sempre menor que a unidade, tipicamente variando entre 0,95 e 0,99 para medidores de Venturi bem projetados. Este coeficiente depende de fatores como o número de Reynolds, a razão de diâmetros $\beta = D_2/D_1$, a rugosidade da superfície interna e a geometria específica do medidor.

A equação do medidor de Venturi real, considerando o coeficiente de descarga, é expressa como:

\begin{equation}
Q = C_d A_2 \sqrt{\frac{2\Delta P}{\rho(1 - \beta^4)}}
\label{eq:venturi_real_vazao}
\end{equation}

onde $\beta = D_2/D_1$ é a razão entre o diâmetro da garganta e o diâmetro de entrada, e $A_2$ é a área da seção da garganta.

Na prática, a diferença de pressão é frequentemente medida utilizando um manômetro diferencial em U, que contém um fluido manométrico de densidade $\rho_m$ diferente da densidade do fluido em escoamento. O desnível $\Delta h$ no manômetro está relacionado à diferença de pressão através da equação:

\begin{equation}
\Delta h = \frac{\Delta P}{(\rho_m - \rho)g}
\label{eq:manometro}
\end{equation}

onde $\rho_m$ é a densidade do fluido manométrico em quilogramas por metro cúbico (\si{\kilogram\per\meter\cubed}). Para medidores utilizando mercúrio como fluido manométrico, $\rho_m = 13600$ \si{\kilogram\per\meter\cubed}.

Combinando a equação do medidor com a relação do manômetro, obtém-se a expressão para calcular a vazão a partir do desnível manométrico medido:

\begin{equation}
Q = C_d A_2 \sqrt{\frac{2g \Delta h (\rho_m - \rho)}{\rho(1 - \beta^4)}}
\label{eq:venturi_desnivel}
\end{equation}

Esta equação é fundamental para o modo medidor da aplicação, onde o usuário fornece o desnível manométrico e o sistema calcula a vazão correspondente.

Para considerar as perdas de energia por atrito no escoamento real, utiliza-se a equação de Darcy-Weisbach, que relaciona a perda de carga $h_L$ com as características do escoamento e da tubulação:

\begin{equation}
h_L = f \frac{L}{D} \frac{v^2}{2g}
\label{eq:darcy_weisbach}
\end{equation}

onde $f$ é o coeficiente de atrito de Darcy-Weisbach (adimensional), $L$ é o comprimento do trecho considerado em metros (\si{\meter}), $D$ é o diâmetro hidráulico em metros (\si{\meter}), e $v$ é a velocidade média do escoamento em metros por segundo (\si{\meter\per\second}). O coeficiente de atrito depende do número de Reynolds e da rugosidade relativa da parede do tubo.

A equação de Bernoulli modificada, incluindo as perdas de carga, é expressa como:

\begin{equation}
\frac{P_1}{\rho g} + \frac{v_1^2}{2g} + z_1 = \frac{P_2}{\rho g} + \frac{v_2^2}{2g} + z_2 + h_L
\label{eq:bernoulli_perdas}
\end{equation}

O número de Reynolds é um parâmetro adimensional que caracteriza o regime de escoamento, sendo definido como:

\begin{equation}
\text{Re} = \frac{\rho v D}{\mu} = \frac{v D}{\nu}
\label{eq:reynolds}
\end{equation}

onde $\mu$ é a viscosidade dinâmica em pascais por segundo (\si{\pascal\second}) e $\nu$ é a viscosidade cinemática em metros quadrados por segundo (\si{\meter\squared\per\second}). Para escoamentos em tubos, o regime laminar ocorre para $\text{Re} < 2300$, o regime turbulento para $\text{Re} > 4000$, e a região intermediária é denominada de transição.

\subsection{Arquitetura e Implementação da Aplicação}

A aplicação foi desenvolvida utilizando a linguagem de programação Python, aproveitando suas bibliotecas científicas e de visualização para implementar os cálculos e gerar as visualizações gráficas. A interface web foi construída utilizando o framework Streamlit, que permite criar aplicações interativas de forma eficiente, com controles deslizantes para ajuste de parâmetros e atualização automática dos resultados.

A estrutura da aplicação é organizada em módulos especializados, seguindo princípios de programação orientada a objetos e separação de responsabilidades. O módulo principal \texttt{app.py} contém a interface Streamlit e a lógica de apresentação, enquanto os cálculos são implementados na classe \texttt{VenturiSimulator} no módulo \texttt{app\_modules/simulator.py}. As funções de visualização gráfica são organizadas no módulo \texttt{app\_modules/plots.py}, e os exemplos práticos são implementados no módulo \texttt{app\_modules/examples.py}.

A classe \texttt{VenturiSimulator} encapsula toda a lógica de cálculo do medidor de Venturi. O método principal \texttt{calcular()} recebe todos os parâmetros de entrada necessários e, dependendo do modo de operação selecionado, chama métodos específicos para realizar os cálculos apropriados. Para o modo medidor, o método \texttt{\_calcular\_vazao\_de\_desnivel()} implementa a equação~\eqref{eq:venturi_desnivel} para calcular a vazão a partir do desnível manométrico fornecido. Para os modos ideal e realista, o método \texttt{\_calcular\_desnivel\_de\_vazao()} implementa os cálculos inversos, determinando o desnível manométrico a partir da vazão especificada.

No modo ideal, a aplicação utiliza a equação~\eqref{eq:delta_p_ideal} para calcular a queda de pressão, assumindo que não há perdas de energia, ou seja, $h_L = 0$. No modo realista, o método \texttt{\_calcular\_perda\_carga()} implementa a equação de Darcy-Weisbach para calcular as perdas, e a queda de pressão é ajustada de acordo com a equação de Bernoulli modificada, considerando que parte da energia mecânica é dissipada por atrito.

A implementação do cálculo de perda de carga utiliza uma velocidade média entre as velocidades nas seções de entrada e garganta, e um diâmetro médio entre os diâmetros destas seções, fornecendo uma aproximação razoável para o comportamento do escoamento ao longo do medidor. Esta abordagem simplificada é adequada para fins educacionais, embora em aplicações mais precisas seja necessário considerar o perfil de velocidade ao longo de todo o comprimento do medidor.

As áreas das seções transversais são calculadas utilizando a fórmula da área de um círculo, $A = \pi D^2/4$, onde $D$ é o diâmetro da seção. As velocidades são então determinadas a partir da equação da continuidade, $v = Q/A$, garantindo que a vazão seja mantida constante ao longo do medidor.

O cálculo do número de Reynolds é implementado no método \texttt{calcular\_reynolds()}, utilizando a viscosidade cinemática da água a 20°C, $\nu = 1 \times 10^{-6}$ \si{\meter\squared\per\second}, como valor padrão. Esta aproximação é razoável para a maioria das aplicações educacionais, embora a viscosidade real dependa da temperatura e da natureza do fluido.

\subsection{Visualizações Gráficas}

A aplicação oferece quatro tipos principais de visualizações gráficas, cada uma projetada para ilustrar aspectos específicos do comportamento do medidor de Venturi. Esta seção descreve cada tipo de visualização e identifica as figuras que devem ser incluídas no relatório para documentar adequadamente as capacidades da aplicação.

A primeira visualização é o diagrama esquemático do medidor de Venturi, que apresenta a geometria do dispositivo com indicação clara das seções de entrada, convergente, garganta e divergente. Este diagrama mostra as posições dos pontos de medição de pressão e indica as velocidades calculadas em cada seção, facilitando a compreensão da geometria do medidor e da localização das tomadas de pressão. A Figura~\ref{fig:diagrama_venturi} deve ser incluída no relatório, mostrando um exemplo típico com diâmetros $D_1 = 0,10$ \si{\meter} e $D_2 = 0,05$ \si{\meter}, com as velocidades $v_1$ e $v_2$ claramente indicadas.

A segunda visualização consiste no diagrama do manômetro diferencial em U, que ilustra visualmente o desnível do fluido manométrico em função da diferença de pressão entre as seções de entrada e garganta. Esta representação é fundamental para compreender a relação entre a medição física do desnível $\Delta h$ e a diferença de pressão $\Delta P$ calculada. A Figura~\ref{fig:manometro} deve ser incluída, mostrando um exemplo com desnível típico de 15 a 20 centímetros, com indicação clara dos pontos de conexão $P_1$ e $P_2$ e do desnível $\Delta h$ medido.

A terceira visualização é o perfil de pressão ao longo do tubo, que mostra como a pressão estática varia desde a entrada até a saída do medidor. Este gráfico evidencia claramente a queda de pressão na seção convergente, a pressão mínima na garganta, e a recuperação parcial de pressão na seção divergente. A Figura~\ref{fig:perfil_pressao_ideal} deve mostrar o perfil no modo ideal, onde a recuperação de pressão é completa, enquanto a Figura~\ref{fig:perfil_pressao_realista} deve mostrar o mesmo perfil no modo realista, evidenciando a perda permanente de pressão devido às perdas por atrito. A comparação entre estas duas figuras ilustra claramente o impacto das perdas de energia no desempenho do medidor.

A quarta visualização são as linhas de energia e piezométrica, que fornecem uma representação completa da distribuição de energia ao longo do medidor. A linha piezométrica representa a carga piezométrica $P/(\rho g) + z$, enquanto a linha de energia representa a energia total por unidade de peso $P/(\rho g) + v^2/(2g) + z$. A distância vertical entre estas linhas representa a carga cinética $v^2/(2g)$, e a diferença entre a linha de energia no início e no final do medidor representa a perda de carga total $h_L$. A Figura~\ref{fig:linhas_energia_ideal} deve mostrar as linhas para o modo ideal, onde a linha de energia permanece horizontal, enquanto a Figura~\ref{fig:linhas_energia_realista} deve mostrar o modo realista, onde a linha de energia decresce devido às perdas de carga, com indicação explícita do valor de $h_L$.

Além destas quatro visualizações principais, a aplicação também gera gráficos auxiliares para análise paramétrica. A Figura~\ref{fig:curva_calibracao} deve mostrar uma curva de calibração típica, relacionando a vazão volumétrica $Q$ com o desnível manométrico $\Delta h$, demonstrando a relação quadrática característica onde $Q \propto \sqrt{\Delta h}$. Esta curva é fundamental para o projeto e calibração de medidores de Venturi em aplicações práticas.

A Figura~\ref{fig:sensibilidade_cd} deve ilustrar a sensibilidade da vazão ao coeficiente de descarga, mostrando como variações em $C_d$ afetam a vazão calculada para um desnível fixo. Esta análise é crucial para compreender a importância da calibração adequada do medidor. Finalmente, a Figura~\ref{fig:efeito_beta} deve mostrar o efeito da razão de diâmetros $\beta = D_2/D_1$ no desnível manométrico e na velocidade na garganta, ilustrando o compromisso entre sensibilidade e perdas de carga que deve ser considerado no projeto do medidor.

Todas as visualizações são geradas utilizando a biblioteca Matplotlib, com estilos e cores cuidadosamente selecionados para facilitar a interpretação. Os gráficos são atualizados automaticamente sempre que os parâmetros são modificados pelo usuário, proporcionando uma experiência interativa imediata. As figuras incluídas no relatório devem ser capturadas da aplicação em execução, utilizando os parâmetros típicos mencionados nos exercícios resolvidos, garantindo consistência entre os cálculos manuais e as visualizações apresentadas.

\subsection{Modos de Operação}

A aplicação implementa três modos distintos de operação, cada um adequado para diferentes tipos de análise. O modo ideal considera o escoamento sem perdas de energia, estabelecendo $C_d = 1,0$ e $h_L = 0$, permitindo a análise do comportamento teórico do medidor sob condições ideais. Este modo é útil para compreender os princípios fundamentais e comparar com o comportamento real.

O modo realista incorpora os efeitos de perdas por atrito e utiliza um coeficiente de descarga menor que a unidade, tipicamente entre 0,95 e 0,98, simulando condições mais próximas da realidade operacional. Neste modo, a equação de Darcy-Weisbach é aplicada para calcular as perdas de carga, e a queda de pressão é ajustada de acordo.

O modo medidor simula o uso prático do dispositivo, onde um operador mede o desnível manométrico $\Delta h$ e deseja determinar a vazão correspondente. Neste modo, a aplicação utiliza a equação~\eqref{eq:venturi_desnivel} para calcular a vazão a partir do desnível fornecido, considerando o coeficiente de descarga especificado.

% ============================================
% RESULTADOS E DISCUSSÃO
% ============================================
\section{Resultados e Discussão}

A aplicação desenvolvida foi testada com diversos conjuntos de parâmetros para validar a precisão dos cálculos e demonstrar suas capacidades analíticas. Os testes abrangeram diferentes configurações geométricas, propriedades de fluidos e condições de escoamento, permitindo uma avaliação abrangente do desempenho da ferramenta.

Para um medidor de Venturi típico com diâmetro de entrada $D_1 = 0,10$ \si{\meter}, diâmetro de garganta $D_2 = 0,05$ \si{\meter}, e vazão $Q = 0,015$ \si{\meter\cubed\per\second}, a aplicação calcula velocidades de $v_1 = 1,91$ \si{\meter\per\second} na entrada e $v_2 = 7,64$ \si{\meter\per\second} na garganta, resultando em uma razão de velocidades $v_2/v_1 = 4,0$, que corresponde exatamente à razão inversa das áreas, conforme esperado pela equação da continuidade.

A queda de pressão calculada para este caso no modo ideal é $\Delta P = 27,35$ \si{\kilo\pascal}, resultando em um desnível manométrico de $\Delta h = 22,1$ \si{\centi\meter} quando utiliza-se mercúrio ($\rho_m = 13600$ \si{\kilogram\per\meter\cubed}) como fluido manométrico. Estes valores são consistentes com os cálculos teóricos baseados nas equações fundamentais apresentadas anteriormente.

A comparação entre os modos ideal e realista revela diferenças significativas nos resultados. Para a mesma vazão e geometria, o modo realista com coeficiente de atrito $f = 0,025$ e coeficiente de descarga $C_d = 0,96$ apresenta uma queda de pressão aproximadamente 5\% maior que o modo ideal, devido às perdas por atrito. A perda de carga calculada é $h_L = 0,0045$ \si{\meter}, representando uma pequena fração da energia total, mas suficiente para demonstrar o efeito das perdas reais no desempenho do medidor.

A análise de sensibilidade ao coeficiente de descarga demonstrou que variações relativamente pequenas neste parâmetro podem resultar em erros significativos na medição de vazão. Para um desnível fixo de $\Delta h = 0,15$ \si{\meter}, uma variação de $C_d$ de 0,90 a 1,00 resulta em uma variação de aproximadamente 11\% na vazão calculada. Esta observação destaca a importância crítica da calibração adequada do medidor e da determinação precisa do coeficiente de descarga para cada configuração específica.

O efeito da razão de diâmetros $\beta = D_2/D_1$ no desempenho do medidor foi analisado mantendo o diâmetro de entrada fixo em $D_1 = 0,10$ \si{\meter} e variando o diâmetro da garganta. Os resultados mostram que, para uma vazão constante, uma razão $\beta$ menor resulta em maior velocidade na garganta e maior queda de pressão, aumentando a sensibilidade do medidor. No entanto, valores muito baixos de $\beta$ podem resultar em perdas de carga excessivas e possíveis problemas de cavitação. A faixa típica recomendada é $0,4 \leq \beta \leq 0,7$, onde o medidor oferece um bom compromisso entre sensibilidade e eficiência energética.

A geração de curvas de calibração para diferentes faixas de vazão demonstra a capacidade da aplicação de auxiliar no projeto e na calibração de medidores de Venturi. A relação entre vazão e desnível manométrico segue uma dependência quadrática, conforme esperado da equação~\eqref{eq:venturi_desnivel}, onde $Q \propto \sqrt{\Delta h}$. Esta relação é claramente visualizada nos gráficos gerados pela aplicação, facilitando a compreensão do comportamento do medidor.

O número de Reynolds calculado para os casos testados varia entre $10^4$ e $10^5$, indicando regime turbulento, que é o regime típico de operação para medidores de Venturi em aplicações industriais. Para números de Reynolds acima de $10^5$, o coeficiente de descarga tende a estabilizar-se, tornando o medidor menos sensível a variações nas condições de escoamento.

A precisão dos cálculos implementados foi validada comparando os resultados com soluções analíticas das equações fundamentais e com dados de literatura técnica. Os valores calculados apresentam concordância excelente com os resultados teóricos esperados, com diferenças menores que 0,1\% para os casos testados no modo ideal, confirmando a correta implementação das equações fundamentais.

A interface interativa da aplicação permite exploração rápida de diferentes cenários, facilitando a compreensão de como variações nos parâmetros geométricos, propriedades dos fluidos e condições de escoamento afetam o desempenho do medidor. Esta capacidade de análise paramétrica é valiosa tanto para fins educacionais quanto para aplicações práticas de projeto e análise.

% ============================================
% EXERCÍCIOS RESOLVIDOS
% ============================================
\section{Exercícios Resolvidos para Validação do Simulador}

Esta seção apresenta exercícios práticos resolvidos passo a passo, permitindo que os estudantes realizem cálculos manuais e comparem os resultados com os valores obtidos através do simulador. Esta abordagem valida tanto a precisão do simulador quanto a compreensão dos conceitos teóricos pelos estudantes.

\subsection{Exercício 1: Cálculo de Vazão a partir do Desnível Manométrico}

\textbf{Enunciado:} Um medidor de Venturi possui diâmetro de entrada $D_1 = 0,10$ \si{\meter} e diâmetro de garganta $D_2 = 0,05$ \si{\meter}. O medidor está instalado em uma tubulação horizontal transportando água ($\rho = 1000$ \si{\kilogram\per\meter\cubed}) e utiliza mercúrio ($\rho_m = 13600$ \si{\kilogram\per\meter\cubed}) como fluido manométrico. Um manômetro diferencial em U conectado ao medidor indica um desnível de $\Delta h = 0,15$ \si{\meter}. Considerando um coeficiente de descarga $C_d = 0,98$, determine a vazão volumétrica do escoamento.

\textbf{Resolução:}

Primeiramente, calcula-se a razão de diâmetros $\beta$:

\begin{equation}
\beta = \frac{D_2}{D_1} = \frac{0,05}{0,10} = 0,5
\end{equation}

A área da seção da garganta é calculada como:

\begin{equation}
A_2 = \frac{\pi D_2^2}{4} = \frac{\pi (0,05)^2}{4} = 0,001963 \si{\meter\squared}
\end{equation}

Aplicando a equação~\eqref{eq:venturi_desnivel} para calcular a vazão:

\begin{align}
Q &= C_d A_2 \sqrt{\frac{2g \Delta h (\rho_m - \rho)}{\rho(1 - \beta^4)}}\\
Q &= 0,98 \times 0,001963 \times \sqrt{\frac{2 \times 9,81 \times 0,15 \times (13600 - 1000)}{1000 \times (1 - 0,5^4)}}\\
Q &= 0,001923 \times \sqrt{\frac{2 \times 9,81 \times 0,15 \times 12600}{1000 \times (1 - 0,0625)}}\\
Q &= 0,001923 \times \sqrt{\frac{37062,6}{937,5}}\\
Q &= 0,001923 \times \sqrt{39,533}\\
Q &= 0,001923 \times 6,288\\
Q &= 0,0121 \si{\meter\cubed\per\second} = 12,1 \si{\liter\per\second}
\end{align}

\textbf{Validação no Simulador:} Configure o simulador no modo Medidor com os parâmetros: $D_1 = 0,10$ \si{\meter}, $D_2 = 0,05$ \si{\meter}, $\rho = 1000$ \si{\kilogram\per\meter\cubed}, $\rho_m = 13600$ \si{\kilogram\per\meter\cubed}, $\Delta h = 0,15$ \si{\meter}, e $C_d = 0,98$. O simulador deve retornar uma vazão de aproximadamente $Q = 0,0121$ \si{\meter\cubed\per\second}, validando o cálculo manual.

\subsection{Exercício 2: Cálculo do Desnível Manométrico para uma Vazão Conhecida}

\textbf{Enunciado:} Para o mesmo medidor do exercício anterior, determine o desnível manométrico esperado quando a vazão é $Q = 0,015$ \si{\meter\cubed\per\second}. Considere escoamento ideal, sem perdas de energia.

\textbf{Resolução:}

Primeiramente, calculam-se as áreas das seções:

\begin{align}
A_1 &= \frac{\pi D_1^2}{4} = \frac{\pi (0,10)^2}{4} = 0,007854 \si{\meter\squared}\\
A_2 &= \frac{\pi D_2^2}{4} = \frac{\pi (0,05)^2}{4} = 0,001963 \si{\meter\squared}
\end{align}

As velocidades nas seções são determinadas pela equação da continuidade:

\begin{align}
v_1 &= \frac{Q}{A_1} = \frac{0,015}{0,007854} = 1,910 \si{\meter\per\second}\\
v_2 &= \frac{Q}{A_2} = \frac{0,015}{0,001963} = 7,641 \si{\meter\per\second}
\end{align}

Para escoamento ideal, a queda de pressão é calculada pela equação~\eqref{eq:delta_p_ideal}:

\begin{align}
\Delta P &= \frac{1}{2}\rho(v_2^2 - v_1^2)\\
\Delta P &= \frac{1}{2} \times 1000 \times (7,641^2 - 1,910^2)\\
\Delta P &= 500 \times (58,39 - 3,648)\\
\Delta P &= 500 \times 54,74\\
\Delta P &= 27370 \si{\pascal} = 27,37 \si{\kilo\pascal}
\end{align}

O desnível manométrico é então calculado pela equação~\eqref{eq:manometro}:

\begin{align}
\Delta h &= \frac{\Delta P}{(\rho_m - \rho)g}\\
\Delta h &= \frac{27370}{(13600 - 1000) \times 9,81}\\
\Delta h &= \frac{27370}{123606}\\
\Delta h &= 0,221 \si{\meter} = 22,1 \si{\centi\meter}
\end{align}

\textbf{Validação no Simulador:} Configure o simulador no modo Ideal com $D_1 = 0,10$ \si{\meter}, $D_2 = 0,05$ \si{\meter}, $\rho = 1000$ \si{\kilogram\per\meter\cubed}, $\rho_m = 13600$ \si{\kilogram\per\meter\cubed}, e $Q = 0,015$ \si{\meter\cubed\per\second}. O simulador deve retornar um desnível de aproximadamente $\Delta h = 0,221$ \si{\meter}, confirmando o cálculo teórico.

\subsection{Exercício 3: Comparação entre Modo Ideal e Modo Realista}

\textbf{Enunciado:} Para o medidor dos exercícios anteriores, com vazão $Q = 0,015$ \si{\meter\cubed\per\second}, compare os resultados obtidos no modo ideal e no modo realista, considerando comprimento do medidor $L = 1,0$ \si{\meter}, coeficiente de atrito $f = 0,025$, e coeficiente de descarga $C_d = 0,96$.

\textbf{Resolução:}

No modo ideal, conforme calculado no exercício anterior, tem-se $\Delta P = 27,37$ \si{\kilo\pascal} e $\Delta h = 22,1$ \si{\centi\meter}, sem perdas de carga ($h_L = 0$).

Para o modo realista, primeiramente calcula-se a perda de carga utilizando a equação de Darcy-Weisbach. A velocidade média é calculada como a média aritmética das velocidades:

\begin{equation}
v_{\text{média}} = \frac{v_1 + v_2}{2} = \frac{1,910 + 7,641}{2} = 4,776 \si{\meter\per\second}
\end{equation}

O diâmetro médio é calculado como:

\begin{equation}
D_{\text{média}} = \frac{D_1 + D_2}{2} = \frac{0,10 + 0,05}{2} = 0,075 \si{\meter}
\end{equation}

Aplicando a equação~\eqref{eq:darcy_weisbach}:

\begin{align}
h_L &= f \frac{L}{D_{\text{média}}} \frac{v_{\text{média}}^2}{2g}\\
h_L &= 0,025 \times \frac{1,0}{0,075} \times \frac{4,776^2}{2 \times 9,81}\\
h_L &= 0,025 \times 13,33 \times \frac{22,81}{19,62}\\
h_L &= 0,025 \times 13,33 \times 1,162\\
h_L &= 0,387 \si{\meter}
\end{align}

A queda de pressão no modo realista é ajustada considerando as perdas:

\begin{align}
\Delta P_{\text{realista}} &= \rho\left[\frac{1}{2}(v_2^2 - v_1^2) + g h_L\right]\\
\Delta P_{\text{realista}} &= 1000 \times \left[\frac{1}{2}(58,39 - 3,648) + 9,81 \times 0,387\right]\\
\Delta P_{\text{realista}} &= 1000 \times (27,37 + 3,80)\\
\Delta P_{\text{realista}} &= 31170 \si{\pascal} = 31,17 \si{\kilo\pascal}
\end{align}

O desnível manométrico no modo realista é:

\begin{align}
\Delta h_{\text{realista}} &= \frac{31170}{(13600 - 1000) \times 9,81}\\
\Delta h_{\text{realista}} &= \frac{31170}{123606}\\
\Delta h_{\text{realista}} &= 0,252 \si{\meter} = 25,2 \si{\centi\meter}
\end{align}

A diferença percentual no desnível é:

\begin{equation}
\text{Diferença} = \frac{25,2 - 22,1}{22,1} \times 100\% = 14,0\%
\end{equation}

\textbf{Validação no Simulador:} Configure o simulador primeiro no modo Ideal e depois no modo Realista com os mesmos parâmetros: $D_1 = 0,10$ \si{\meter}, $D_2 = 0,05$ \si{\meter}, $L = 1,0$ \si{\meter}, $\rho = 1000$ \si{\kilogram\per\meter\cubed}, $\rho_m = 13600$ \si{\kilogram\per\meter\cubed}, $Q = 0,015$ \si{\meter\cubed\per\second}, $f = 0,025$, e $C_d = 0,96$. Compare os valores de $\Delta h$ e $\Delta P$ entre os dois modos, verificando que o modo realista apresenta valores maiores devido às perdas de energia.

\subsection{Exercício 4: Análise da Influência da Razão de Diâmetros}

\textbf{Enunciado:} Um medidor de Venturi possui diâmetro de entrada fixo $D_1 = 0,10$ \si{\meter} e vazão constante $Q = 0,015$ \si{\meter\cubed\per\second}. Determine o desnível manométrico para três valores diferentes do diâmetro da garganta: $D_2 = 0,03$ \si{\meter}, $D_2 = 0,05$ \si{\meter}, e $D_2 = 0,07$ \si{\meter}. Considere escoamento ideal com água e mercúrio como fluido manométrico.

\textbf{Resolução:}

Para cada valor de $D_2$, calcula-se a razão $\beta$, a área da garganta, as velocidades e o desnível. Os cálculos seguem o mesmo procedimento do exercício 2. Os resultados são apresentados na Tabela~\ref{tab:exercicio4}.

\begin{table}[H]
\centering
\caption{Resultados para diferentes razões de diâmetros}
\label{tab:exercicio4}
\begin{tabular}{ccccc}
\toprule
$D_2$ (\si{\meter}) & $\beta$ & $v_2$ (\si{\meter\per\second}) & $\Delta P$ (\si{\kilo\pascal}) & $\Delta h$ (\si{\centi\meter}) \\
\midrule
0,03 & 0,3 & 21,22 & 220,8 & 178,5 \\
0,05 & 0,5 & 7,64 & 27,4 & 22,1 \\
0,07 & 0,7 & 3,90 & 7,1 & 5,7 \\
\bottomrule
\end{tabular}
\end{table}

Os resultados demonstram claramente que, para uma vazão constante, uma razão $\beta$ menor resulta em maior velocidade na garganta e consequentemente maior queda de pressão e desnível manométrico. Este comportamento aumenta a sensibilidade do medidor, mas também aumenta as perdas de carga.

\textbf{Validação no Simulador:} Configure o simulador no modo Ideal e teste cada um dos três valores de $D_2$, mantendo os demais parâmetros constantes. Compare os valores de $\Delta h$ obtidos com os valores calculados na tabela, verificando a concordância entre os resultados teóricos e os valores do simulador.

% ============================================
% CONCLUSÃO
% ============================================
\section{Conclusão}

O desenvolvimento desta aplicação web interativa para simulação de medidores de Venturi demonstrou ser uma ferramenta eficaz para o ensino e a compreensão de conceitos fundamentais de mecânica dos fluidos relacionados a medição de vazão. A implementação das equações fundamentais, incluindo equação de Bernoulli, equação da continuidade e equação de Darcy-Weisbach, resultou em cálculos precisos que concordam com soluções analíticas e dados de literatura técnica.

A disponibilização de três modos distintos de operação permite aos usuários comparar o comportamento ideal e real do medidor, facilitando a compreensão do impacto das perdas de energia e da importância do coeficiente de descarga na precisão das medições. As visualizações gráficas desenvolvidas proporcionam uma representação clara e intuitiva dos fenômenos físicos envolvidos, complementando efetivamente a apresentação dos resultados numéricos.

Os resultados obtidos através da aplicação revelaram aspectos importantes do comportamento dos medidores de Venturi, incluindo a sensibilidade crítica ao coeficiente de descarga, a influência da razão de diâmetros no desempenho do medidor, e a relação quadrática entre vazão e desnível manométrico. Estas observações são consistentes com a teoria estabelecida e demonstram a validade da implementação.

A análise de sensibilidade realizada através da aplicação destacou que pequenas variações no coeficiente de descarga podem resultar em erros significativos na medição de vazão, enfatizando a importância da calibração adequada dos medidores para aplicações que requerem alta precisão. A capacidade da ferramenta de gerar curvas de calibração para diferentes configurações geométricas e condições operacionais demonstra seu potencial utilitário além do ambiente educacional.

As principais observações deste trabalho incluem a validação da abordagem de implementação das equações fundamentais, a eficácia da interface interativa para exploração paramétrica, e a utilidade das visualizações gráficas para compreensão conceitual. A aplicação desenvolvida serve tanto como ferramenta educacional para estudantes de engenharia quanto como instrumento de análise técnica para profissionais envolvidos no projeto e operação de sistemas de medição de vazão.

Os aprendizados obtidos durante o desenvolvimento incluem a importância de uma estrutura modular bem organizada para facilitar manutenção e extensão do código, a necessidade de validação cuidadosa dos cálculos através de comparação com soluções analíticas conhecidas, e o valor de uma interface intuitiva para maximizar a utilidade da ferramenta.

Possíveis melhorias futuras para o projeto incluem a incorporação de correções mais sofisticadas para o cálculo de perdas de carga, considerando o perfil completo de velocidade ao longo do medidor; a implementação de correções para efeitos de temperatura e viscosidade sobre as propriedades dos fluidos; a adição de análises de incerteza e propagação de erros; e a extensão para incluir outros tipos de medidores de vazão, como placas de orifício e bocais, permitindo comparações diretas entre diferentes dispositivos.

A aplicação desenvolvida representa uma contribuição valiosa para o ensino de mecânica dos fluidos e para a análise técnica de sistemas de medição de vazão, demonstrando o potencial das ferramentas computacionais interativas como complemento ao ensino tradicional e como instrumentos práticos de engenharia.

% ============================================
% REFERÊNCIAS BIBLIOGRÁFICAS
% ============================================
\section{Referências Bibliográficas}

\begin{thebibliography}{99}

\bibitem{munson2004}
MUNSON, Bruce R.; YOUNG, Donald F.; OKIISHI, Theodore H. \textbf{Fundamentos da Mecânica dos Fluidos}. Tradução da 4ª edição americana. São Paulo: Editora Edgard Blücher, 2004.

\bibitem{fox2010}
FOX, Robert W.; MCDONALD, Alan T.; PRITCHARD, Philip J. \textbf{Introdução à Mecânica dos Fluidos}. 7ª ed. Rio de Janeiro: LTC, 2010.

\bibitem{white2011}
WHITE, Frank M. \textbf{Fluid Mechanics}. 7th ed. New York: McGraw-Hill, 2011.

\bibitem{cengel2015}
ÇENGEL, Yunus A.; CIMBALA, John M. \textbf{Mecânica dos Fluidos: Fundamentos e Aplicações}. 3ª ed. Porto Alegre: AMGH Editora, 2015.

\bibitem{iso5167}
INTERNATIONAL ORGANIZATION FOR STANDARDIZATION. \textbf{ISO 5167-4: Measurement of fluid flow by means of pressure differential devices -- Part 4: Venturi tubes}. Geneva: ISO, 2003.

\bibitem{miller1996}
MILLER, Richard W. \textbf{Flow Measurement Engineering Handbook}. 3rd ed. New York: McGraw-Hill, 1996.

\bibitem{streeter1991}
STREETER, Victor L.; WYLIE, E. Benjamin; BEDFORD, Keith W. \textbf{Mecânica dos Fluidos}. 9ª ed. São Paulo: McGraw-Hill, 1991.

\bibitem{streamlit}
STREAMLIT. \textbf{Streamlit Documentation}. Disponível em: \url{https://docs.streamlit.io/}. Acesso em: 2024.

\end{thebibliography}

\end{document}

