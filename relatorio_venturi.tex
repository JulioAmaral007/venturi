\documentclass[12pt,a4paper]{article}
\usepackage[utf8]{inputenc}
\usepackage[portuguese]{babel}
\usepackage{amsmath}
\usepackage{geometry}
\usepackage{siunitx}
\usepackage{booktabs}
\usepackage{float}
\usepackage{hyperref}

\geometry{margin=2cm}
\setlength{\parskip}{0.3em}

\title{\textbf{Relatório Técnico: Projeto de Medidor de Venturi}}
\author{Simulador Interativo de Medidor de Venturi}
\date{\today}

\begin{document}

\maketitle

\section{Introdução}

\subsection{O que é um Tubo de Venturi}

O medidor de Venturi é um dispositivo de medição de vazão baseado no princípio de Bernoulli, inventado por Giovanni Battista Venturi no século XVIII. Consiste em um tubo com três seções: \textbf{convergente} (redução gradual), \textbf{garganta} (diâmetro mínimo D₂) e \textbf{divergente} (recuperação gradual).

O princípio de funcionamento: quando o fluido passa pela garganta, sua velocidade aumenta e a pressão diminui. A diferença de pressão entre entrada (P₁) e garganta (P₂) é proporcional à vazão.

\subsection{Objetivo do Projeto}

Desenvolver um simulador interativo para: calcular vazões a partir de medições de pressão, analisar escoamento em diferentes condições (ideal, realista, medidor), visualizar perfis de pressão e energia, gerar curvas de calibração, e estudar influência de parâmetros (Cd, Reynolds).

\subsection{Áreas de Aplicação}

Mecânica dos fluidos, medições de vazão industrial, tratamento de água, indústria petrolífera, química/petroquímica, geração de energia e mineração.

\section{Fundamentos Teóricos}

\subsection{Equação da Continuidade}

Expressa conservação de massa para fluido incompressível em regime permanente:

\begin{equation}
A_1 v_1 = A_2 v_2 = Q
\label{eq:continuidade}
\end{equation}

Onde: $A_1, A_2$ = áreas (m²), $v_1, v_2$ = velocidades (m/s), $Q$ = vazão volumétrica (m³/s).

\subsection{Equação de Bernoulli}

Para tubo horizontal (z₁ = z₂), relaciona pressão e velocidade:

\begin{equation}
P_1 + \frac{1}{2}\rho v_1^2 = P_2 + \frac{1}{2}\rho v_2^2
\label{eq:bernoulli}
\end{equation}

Onde: $P_1, P_2$ = pressões (Pa), $\rho$ = densidade (kg/m³), $v_1, v_2$ = velocidades (m/s).

Aplica-se para escoamento permanente, incompressível e ideal (sem perdas).

\subsection{Diferença de Pressão}

Combinando continuidade e Bernoulli:

\begin{equation}
\Delta P = P_1 - P_2 = \frac{1}{2}\rho(v_2^2 - v_1^2) = \frac{1}{2}\rho Q^2\left(\frac{1}{A_2^2} - \frac{1}{A_1^2}\right)
\label{eq:delta_p}
\end{equation}

\subsection{Relação Pressão-Velocidade-Área}

Definindo razão de diâmetros $\beta = D_2/D_1$:

\begin{equation}
v_2 = v_1 \left(\frac{D_1}{D_2}\right)^2 = \frac{v_1}{\beta^2}, \quad \Delta P = \frac{1}{2}\rho v_1^2\left(\frac{1}{\beta^4} - 1\right)
\label{eq:beta}
\end{equation}

Quanto menor $\beta$, maior a diferença de pressão para mesma vazão.

\subsection{Cálculo de Vazão}

Vazão volumétrica: $Q = A_2 v_2 = A_1 v_1$. Vazão mássica: $\dot{m} = \rho Q$.

Vazão a partir de $\Delta P$ (ideal): $Q = A_2 \sqrt{2\Delta P / [\rho(1 - \beta^4)]}$

Vazão real (com $C_d$): $Q = C_d A_2 \sqrt{2\Delta P / [\rho(1 - \beta^4)]}$

Vazão a partir de $\Delta h$ (mais usada na prática):
\begin{equation}
Q = C_d A_2 \sqrt{\frac{2g\Delta h(\rho_m - \rho)}{\rho(1 - \beta^4)}}
\label{eq:vazao_delta_h}
\end{equation}

Onde: $C_d$ = coeficiente de descarga (0,95-0,99), $\rho_m$ = densidade do fluido manométrico (kg/m³).

\subsection{Número de Reynolds}

Caracteriza o regime de escoamento:

\begin{equation}
Re = \frac{\rho v D}{\mu} = \frac{v D}{\nu}
\label{eq:reynolds}
\end{equation}

Onde: $\mu$ = viscosidade dinâmica (Pa·s), $\nu$ = viscosidade cinemática (m²/s), $D$ = diâmetro característico (m).

\textbf{Regimes:} Laminar ($Re < 2300$), Transição ($2300 < Re < 4000$), Turbulento ($Re > 4000$). Para medidores, geralmente $Re > 10^4$ é desejável.

\subsection{Perdas de Carga}

No escoamento real, há perdas por atrito (fórmula de Darcy-Weisbach):

\begin{equation}
h_L = f \frac{L}{D} \frac{v^2}{2g}
\label{eq:perdas}
\end{equation}

Equação de Bernoulli modificada: $\frac{P_1}{\rho g} + \frac{v_1^2}{2g} = \frac{P_2}{\rho g} + \frac{v_2^2}{2g} + h_L$

\section{Aplicação das Fórmulas no Projeto}

\subsection{Implementação}

\textbf{Continuidade:} $v_1 = Q/A_1$, $v_2 = Q/A_2$ (áreas: $A = \pi D^2/4$)

\textbf{Modo Ideal:} $\Delta P = \frac{1}{2}\rho(v_2^2 - v_1^2)$, $h_L = 0$

\textbf{Modo Realista:} Inclui perdas: $\Delta P = \rho[\frac{1}{2}(v_2^2 - v_1^2) + g h_L]$

\textbf{Modo Medidor:} Calcula $Q$ a partir de $\Delta h$ usando equação \eqref{eq:vazao_delta_h}

\textbf{Reynolds:} $Re = v_1 D_1 / \nu$ (com $\nu = 10^{-6}$ m²/s para água a 20°C)

\subsection{Exemplo Numérico}

Para $D_1 = 0,10$ m, $D_2 = 0,05$ m, $Q = 0,015$ m³/s, $\rho = 1000$ kg/m³:

\begin{align*}
A_1 &= 0,007854 \text{ m²}, \quad A_2 = 0,001963 \text{ m²} \\
v_1 &= 1,91 \text{ m/s}, \quad v_2 = 7,64 \text{ m/s} \\
\Delta P &= 27,3 \text{ kPa} \quad (\text{modo ideal}) \\
Re &= 191000 \quad (\text{turbulento})
\end{align*}

\section{Exemplos Práticos do Projeto}

\subsection{Exemplo 1: Comparação Ideal vs Realista}

Demonstra diferença entre comportamento teórico e real. Para $D_1 = 0,10$ m, $D_2 = 0,05$ m, $Q = 0,015$ m³/s: Modo Ideal ($C_d = 1,0$) resulta em $\Delta h = 20,8$ cm, enquanto Modo Realista ($C_d = 0,96$) resulta em $\Delta h = 21,7$ cm ($\approx 4,3\%$ maior devido às perdas).

\subsection{Exemplo 2: Curva de Calibração}

Relaciona vazão ($Q$) com desnível ($\Delta h$). Variação de $Q$ de 5 a 30 L/s mostra relação quadrática $Q \propto \sqrt{\Delta h}$, com $\Delta h$ variando de 2,3 a 83 cm.

\subsection{Exemplo 3: Modo Medidor ($\Delta h \to Q$)}

Calcula vazão a partir da leitura do manômetro. Resultados: $\Delta h = 5, 10, 15, 20, 25$ cm correspondem a $Q = 8,45; 11,95; 14,64; 16,90; 18,90$ L/s. Confirma $Q \propto \sqrt{\Delta h}$.

\subsection{Exemplo 4: Sensibilidade ao Coeficiente de Descarga}

Para $\Delta h = 0,15$ m fixo: $C_d = 0,90 \to Q = 13,44$ L/s; $C_d = 1,00 \to Q = 14,94$ L/s. Variação de $10\%$ em $C_d$ causa $11,2\%$ de variação em $Q$, demonstrando importância de $C_d$ preciso (típico: 0,95-0,98).

\subsection{Exemplo 5: Análise de Número de Reynolds}

Identifica regimes: Laminar ($Re < 2300$) - instável; Transição ($2300 < Re < 4000$) - imprevisível; Turbulento ($Re > 4000$) - estável, recomendado. Prática: $Re > 10^4$ desejável, $Re > 2 \times 10^4$ para calibrados (ISO 5167).

\section{Discussão dos Resultados}

\subsection{Conformidade com a Teoria}

Resultados confirmam: continuidade ($A_1 v_1 = A_2 v_2$), Bernoulli, relação $Q \propto \sqrt{\Delta h}$, e Reynolds dentro das faixas esperadas.

\subsection{Discrepâncias e Explicações}

\textbf{Perdas de carga:} Atrito causa aumento de $\Delta P$ (2-5\% típico), modeladas por Darcy-Weisbach.

\textbf{Coeficiente de descarga:} $C_d < 1$ devido a perdas, separação de escoamento, efeitos de borda. Típico: 0,95-0,98. Depende de $Re$, $\beta$ e rugosidade.

\textbf{Erros experimentais:} Precisão do manômetro, variações de temperatura, instalação inadequada, bolhas de ar.

\subsection{Precisão da Medição}

Fatores: incerteza em $C_d$ ($\pm 0,5\%$ a $\pm 1\%$), precisão do manômetro ($\pm 0,1\%$ a $\pm 1\%$), dimensões ($\pm 0,05\%$ para calibrado), propriedades do fluido.

Incerteza total: $\pm 0,5\%$ a $\pm 1,5\%$ (pode chegar a $\pm 0,25\%$ com calibração cuidadosa). Vantagens: baixa perda de carga (80-95\% recuperação), alta precisão, estabilidade, robustez, baixa manutenção.

\section{Exercícios Resolvidos}

Esta seção apresenta exercícios práticos resolvidos passo a passo, permitindo validar os cálculos manuais com os resultados do simulador.

\subsection{Exercício 1: Cálculo de Vazão a partir de $\Delta h$}

\textbf{Enunciado:} Medidor com $D_1 = 0,10$ m, $D_2 = 0,05$ m, água ($\rho = 1000$ kg/m³), mercúrio ($\rho_m = 13600$ kg/m³), $\Delta h = 0,15$ m, e $C_d = 0,98$. Determine $Q$.

\textbf{Resolução:} $\beta = D_2/D_1 = 0,5$, $A_2 = \pi D_2^2/4 = 0,001963$ m². Aplicando a equação~\eqref{eq:vazao_delta_h}:

\begin{align}
Q &= 0,98 \times 0,001963 \times \sqrt{\frac{2 \times 9,81 \times 0,15 \times (13600-1000)}{1000 \times (1-0,5^4)}}\\
Q &= 0,001923 \times \sqrt{\frac{37062,6}{937,5}} = 0,001923 \times 6,288\\
Q &= 0,0121 \text{ m³/s} = 12,1 \text{ L/s}
\end{align}

\textbf{Validação no Simulador:} Configure no modo Medidor: $D_1 = 0,10$ m, $D_2 = 0,05$ m, $\rho = 1000$ kg/m³, $\rho_m = 13600$ kg/m³, $\Delta h = 0,15$ m, $C_d = 0,98$. Resultado esperado: $Q \approx 0,0121$ m³/s.

\subsection{Exercício 2: Cálculo de $\Delta h$ para Vazão Conhecida}

\textbf{Enunciado:} Para o mesmo medidor, determine $\Delta h$ quando $Q = 0,015$ m³/s (modo ideal).

\textbf{Resolução:} $A_1 = 0,007854$ m², $A_2 = 0,001963$ m². Pela continuidade: $v_1 = Q/A_1 = 1,910$ m/s, $v_2 = Q/A_2 = 7,641$ m/s. Aplicando a equação~\eqref{eq:delta_p}:

\begin{align}
\Delta P &= \frac{1}{2} \times 1000 \times (7,641^2 - 1,910^2) = 27370 \text{ Pa}\\
\Delta h &= \frac{27370}{(13600-1000) \times 9,81} = 0,221 \text{ m} = 22,1 \text{ cm}
\end{align}

\textbf{Validação no Simulador:} Configure no modo Ideal: $D_1 = 0,10$ m, $D_2 = 0,05$ m, $\rho = 1000$ kg/m³, $\rho_m = 13600$ kg/m³, $Q = 0,015$ m³/s. Resultado esperado: $\Delta h \approx 0,221$ m.

\subsection{Exercício 3: Comparação Ideal vs Realista}

\textbf{Enunciado:} Compare os resultados para $Q = 0,015$ m³/s, $L = 1,0$ m, $f = 0,025$, $C_d = 0,96$.

\textbf{Resolução:} Modo ideal: $\Delta P = 27,37$ kPa, $\Delta h = 22,1$ cm, $h_L = 0$.

Modo realista: velocidade média $v_{\text{média}} = (1,910 + 7,641)/2 = 4,776$ m/s, diâmetro médio $D_{\text{média}} = 0,075$ m. Aplicando a equação~\eqref{eq:perdas}:

\begin{align}
h_L &= 0,025 \times \frac{1,0}{0,075} \times \frac{4,776^2}{2 \times 9,81} = 0,387 \text{ m}\\
\Delta P_{\text{realista}} &= 1000 \times \left[\frac{1}{2}(7,641^2 - 1,910^2) + 9,81 \times 0,387\right]\\
&= 31170 \text{ Pa} = 31,17 \text{ kPa}\\
\Delta h_{\text{realista}} &= \frac{31170}{123606} = 0,252 \text{ m} = 25,2 \text{ cm}
\end{align}

Diferença: $14,0\%$ no desnível.

\textbf{Validação no Simulador:} Compare os modos Ideal e Realista com os mesmos parâmetros. O modo realista deve apresentar $\Delta h \approx 25,2$ cm.

\subsection{Exercício 4: Efeito da Razão $\beta$}

\textbf{Enunciado:} Para $D_1 = 0,10$ m e $Q = 0,015$ m³/s fixos, determine $\Delta h$ para $D_2 = 0,03$, $0,05$ e $0,07$ m (modo ideal).

\textbf{Resolução:} Para cada $D_2$, calcula-se $\beta$, $A_2$, $v_2$ e $\Delta h$. Resultados:

\begin{table}[h]
\centering
\small
\begin{tabular}{ccccc}
\toprule
$D_2$ (m) & $\beta$ & $v_2$ (m/s) & $\Delta P$ (kPa) & $\Delta h$ (cm) \\
\midrule
0,03 & 0,3 & 21,22 & 220,8 & 178,5 \\
0,05 & 0,5 & 7,64 & 27,4 & 22,1 \\
0,07 & 0,7 & 3,90 & 7,1 & 5,7 \\
\bottomrule
\end{tabular}
\end{table}

Menor $\beta$ resulta em maior sensibilidade, mas também em maiores perdas de carga.

\textbf{Validação no Simulador:} Teste cada valor de $D_2$ no modo Ideal e compare com a tabela.

\section{Conclusão}

\subsection{Resumo}

O projeto demonstrou: validação das equações fundamentais (continuidade, Bernoulli), diferenças entre ideal e real explicadas por perdas e $C_d$, relação $Q \propto \sqrt{\Delta h}$ confirmada, e influência quantificada de $C_d$, $Re$ e $\beta$.

\subsection{Uso do Venturi como Medidor}

Excelente escolha: precisão ($\pm 0,5\%$ a $\pm 1\%$), baixa perda de carga, ampla faixa (4:1 a 10:1), robustez, baixa manutenção, simplicidade.

\subsection{Limitações e Melhorias Futuras}

Limitações: modelo simplificado (perdas aproximadas, $C_d$ fixo), propriedades constantes, geometria simplificada. Melhorias: cálculo dinâmico de $C_d$, modelagem avançada de perdas, propriedades variáveis, análise de incertezas, visualizações 3D, validação experimental.

\end{document}
