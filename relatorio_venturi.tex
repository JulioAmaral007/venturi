\documentclass[12pt]{article}
\usepackage[utf8]{inputenc}
\usepackage[brazil]{babel}
\usepackage{geometry}
\usepackage{amsmath}
\usepackage{amssymb}
\usepackage{hyperref}
\usepackage{enumitem}
\usepackage{booktabs}

\geometry{a4paper, margin=2.5cm}

\title{Relatório Final -- Simulador de Medidor de Venturi}
\author{Disciplina de Mecânica dos Fluidos e Engenharia de Computação}
\date{Novembro de 2025}

\begin{document}
\maketitle

\section{Introdução ao Projeto}
O Simulador de Medidor de Venturi é uma aplicação web interativa desenvolvida em \texttt{Streamlit} para apoiar o ensino de Mecânica dos Fluidos, instrumentação industrial e integração com ferramentas computacionais. O projeto foi concebido para resolver a carência de materiais didáticos que apresentem, de forma visual e parametrizável, os efeitos de variações geométricas, propriedades do fluido e modos de operação (ideal, realista e instrumento) em medidores diferenciais de pressão. A solução permite analisar perdas de carga, número de Reynolds, coeficiente de descarga, curvas de calibração e diferentes regimes de escoamento, agregando valor prático tanto para estudantes quanto para engenheiros de campo.

\section{Descrição Geral do Sistema / Aplicação}
\subsection*{Objetivo e funcionalidades}
O sistema executado por \texttt{streamlit run app.py} oferece:
\begin{itemize}[nosep]
    \item Simulação interativa com configuração direta de diâmetros, comprimentos, vazão ou desnível, coeficiente de descarga (\(C_d\)) e fator de atrito (\(f\)).
    \item Três modos operacionais: \emph{Ideal} (sem perdas), \emph{Realista} (com perdas distribuídas e \(C_d < 1\)) e \emph{Medidor} (cálculo da vazão a partir de \(\Delta h\)).
    \item Visualizações técnicas: diagrama geométrico do Venturi, manômetro diferencial, perfil de pressão, linhas piezométrica e de energia.
    \item Painel de resultados numéricos com velocidade, pressão, vazão, número de Reynolds e perdas de carga.
    \item Módulo de exemplos práticos guiados (\texttt{app\_modules/examples.py}) contendo cinco estudos de caso completos.
\end{itemize}

\subsection*{Arquitetura e fluxo}
\begin{enumerate}[nosep]
    \item \textbf{Interface principal (\texttt{app.py})}: gerencia a barra lateral, coleta entradas do usuário e organiza as abas de visualização. Alterna entre modos de simulação e o painel de exemplos guiados.
    \item \textbf{Módulo de cálculo (\texttt{app\_modules/simulator.py})}: encapsula a classe \texttt{VenturiSimulator}, responsável por calcular velocidades, pressões, vazão, número de Reynolds e perdas de carga utilizando \texttt{NumPy}.
    \item \textbf{Módulo de gráficos (\texttt{app\_modules/plots.py})}: gera figuras em \texttt{Matplotlib} para o diagrama do Venturi, manômetro em U, perfil de pressão e linhas de energia.
    \item \textbf{Módulo de exemplos (\texttt{app\_modules/examples.py})}: organiza os cenários guiados, reaproveitando a classe \texttt{VenturiSimulator} e os gráficos.
    \item \textbf{Configurações complementares}: pasta \texttt{.streamlit/} define temas; \texttt{requirements.txt} lista dependências; \texttt{Venturi.md} documenta a teoria; \texttt{README.md} resume execução e funcionalidades.
\end{enumerate}
Entradas principais são parâmetros geométricos, propriedades dos fluidos e condições de escoamento. As saídas incluem métricas numéricas, gráficos e tabelas exportáveis.

\section{Como a Engenharia de Computação foi utilizada}
O projeto aplica conhecimentos típicos da Engenharia de Computação da seguinte maneira:
\begin{itemize}[nosep]
    \item \textbf{Programação em Python}: estrutura modular, orientação a objetos (classe \texttt{VenturiSimulator}) e scripts interativos.
    \item \textbf{Desenvolvimento de interface}: uso do \texttt{Streamlit} para construir painéis responsivos, abas, métricas e formulários reativos.
    \item \textbf{Processamento de dados}: emprego de \texttt{NumPy} para cálculos vetoriais, \texttt{Pandas} para tabelas de calibração e \texttt{Matplotlib} para visualizações técnicas customizadas.
    \item \textbf{Modelagem matemática e automação}: implementação das equações de Bernoulli, continuidade, Darcy–Weisbach e relações manométricas para calcular pressões, vazões e perdas automaticamente.
    \item \textbf{Raciocínio computacional}: abstração do medidor real em componentes de software, criação de modos idealizados/realistas e geração de cenários de uso replicáveis.
    \item \textbf{Integração com a Mecânica dos Fluidos}: o código converte parâmetros físicos em simulações, permitindo validar hipóteses de fluídos por meio de recursos computacionais acessíveis.
\end{itemize}

\section{Conteúdo de Fluidos aplicado no projeto}

\subsection*{4.1 Definição de fluido, análise e propriedades}
Fluido é qualquer substância capaz de escoar sob pequenas tensões de cisalhamento. Propriedades fundamentais utilizadas: densidade \(\rho\) (kg/m\(^3\)), viscosidade dinâmica \(\mu\) (Pa·s), viscosidade cinemática \(\nu = \mu/\rho\), pressão \(P\) (Pa) e gravidade específica. No simulador, \(\rho\) e \(\rho_m\) (fluido manométrico) são entradas diretas; \(\mu\) é aproximada ao adotar \(\nu = 10^{-6}\,\text{m}^2/\text{s}\) para água em \texttt{VenturiSimulator.calcular\_reynolds}. A modelagem considera regime permanente e fluido incompressível.

\subsection*{4.2 Forças hidrostáticas e empuxo}
As forças hidrostáticas decorrem da distribuição de pressão em fluidos em repouso. A relação entre pressão e coluna manométrica é:
\[
\Delta P = (\rho_m - \rho) g \Delta h
\]
onde \(g=9{,}81\ \text{m/s}^2\) e \(\Delta h\) é o desnível medido. O simulador utiliza essa equação tanto para converter \(\Delta h\) em \(\Delta P\) (modo Medidor) quanto para estimar o desnível a partir da queda de pressão calculada (modos Ideal/Realista). O conceito de empuxo aparece implicitamente ao comparar massas específicas de fluido de processo e manométrico.

\subsection*{4.3 Equações básicas na forma integral para volume de controle}
As bases são a equação da continuidade e a conservação de energia/massa. Para fluido incompressível em regime permanente:
\[
Q = A_1 v_1 = A_2 v_2
\]
e a equação de Bernoulli integral (com perdas) entre duas seções horizontais:
\[
\frac{P_1}{\rho g} + \frac{v_1^2}{2g} = \frac{P_2}{\rho g} + \frac{v_2^2}{2g} + h_L
\]
No código, \texttt{\_calcular\_desnivel\_de\_vazao} combina continuidade para obter \(v_1\) e \(v_2\) e, em seguida, calcula \(\Delta P\) com e sem perdas para alimentar as visualizações e métricas.

\subsection*{4.4 Equações de Euler e Bernoulli}
A equação de Euler descreve o equilíbrio dinâmico em um fluido invíscido. Integrando-a ao longo de uma linha de corrente obtém-se Bernoulli:
\[
\frac{P}{\rho g} + \frac{v^2}{2g} + z = \text{constante}
\]
Para o Venturi horizontal (\(z_1 = z_2\)), resulta:
\[
\Delta P = \frac{1}{2}\rho (v_2^2 - v_1^2)
\]
Para o medidor real, a vazão é corrigida com \(C_d\):
\[
Q = C_d A_2 \sqrt{\frac{2 \Delta P}{\rho (1 - \beta^4)}}, \quad \beta = \frac{D_2}{D_1}.
\]
Essas fórmulas estão implementadas em \texttt{\_calcular\_vazao\_de\_desnivel} e \texttt{\_calcular\_desnivel\_de\_vazao}, permitindo que o aplicativo explore o efeito do coeficiente de descarga e da razão de diâmetros.

\subsection*{4.5 Escoamento em dutos}
O escoamento interno sofre perdas distribuídas representadas pela equação de Darcy–Weisbach:
\[
h_L = f \frac{L}{D_\text{médio}} \frac{v_{\text{médio}}^2}{2g}
\]
Onde \(f\) é controle do usuário, \(L\) é o comprimento efetivo e \(D_\text{médio} = (D_1 + D_2)/2\). O simulador adiciona \(h_L\) à forma modificada de Bernoulli para o modo Realista, produzindo linhas de energia decrescentes e exibindo o impacto de perdas menores e maiores variações de velocidade entre seções.

\section{Funcionamento Técnico do Projeto}
\begin{enumerate}[nosep]
    \item \textbf{Entrada de parâmetros}: a barra lateral coleta geometria (\(D_1\), \(D_2\), \(L\)), propriedades (\(\rho\), \(\rho_m\)), condições de escoamento (vazão \(Q\) ou desnível \(\Delta h\)) e parâmetros avançados (\(f\), \(C_d\)).
    \item \textbf{Seleção do modo}: define se o cálculo parte de \(Q\) (Ideal/Realista) ou de \(\Delta h\) (Medidor). O modo Exemplos desvia para \texttt{executar\_exemplos()}.
    \item \textbf{Processamento}: \texttt{VenturiSimulator.calcular} cria áreas, velocidades, pressões, perdas e número de Reynolds. Modos com perdas invocam \texttt{\_calcular\_perda\_carga}.
    \item \textbf{Visualização}: \texttt{plots.py} gera figuras e \texttt{Streamlit} as apresenta nas abas Diagrama, Manômetro, Pressão, Energia e Resultados.
    \item \textbf{Saída}: métricas (vazão, desnível, velocidades), tabelas exportáveis (\texttt{Pandas DataFrame}) e mensagens de regime (laminar, transição, turbulento).
\end{enumerate}
O fluxo de dados é reativo: qualquer alteração na barra lateral dispara novo cálculo e atualiza todos os componentes simultaneamente, facilitando análises paramétricas.

\section{Exemplos práticos apresentados no projeto}
\begin{description}[style=nextline, leftmargin=0cm]
    \item[Exemplo 1 -- Ideal vs Realista] Compara \(C_d=1\) e \(f=0\) contra \(C_d=0{,}96\) e \(f=0{,}025\) para \(D_1=0{,}10\ \text{m}\), \(D_2=0{,}05\ \text{m}\), \(Q=0{,}015\ \text{m}^3/\text{s}\). O usuário observa o aumento de \(\Delta P\), \(\Delta h\) e \(h_L\) quando há perdas.
    \item[Exemplo 2 -- Curva de calibração] Gera 20 pontos de vazão entre 5 e 30 L/s, exibindo tabela e gráfico \(Q \times \Delta h\). Demonstra a relação \(Q \propto \sqrt{\Delta h}\) e fornece base para calibração de campo.
    \item[Exemplo 3 -- Modo Medidor] Converte leituras de \(\Delta h\) (5 a 25 cm) em vazões e velocidades, além de gráficos \(Q \times \Delta h\) e \(Q \times \sqrt{\Delta h}\). Relaciona teoria de Bernoulli à prática de instrumentação.
    \item[Exemplo 4 -- Sensibilidade ao \(C_d\)] Mantém \(\Delta h = 15\ \text{cm}\) e varia \(C_d\) de 0,90 a 1,00. Mostra que 10\% de variação em \(C_d\) impacta a vazão em magnitude equivalente, alertando para calibração precisa.
    \item[Exemplo 5 -- Número de Reynolds] Varre \(Q\) de 1 a 30 L/s, classifica regimes (laminar, transição, turbulento) e estima \(C_d\) conforme Reynolds. Gráficos explicam a zona recomendada (\(Re > 10^4\)).
\end{description}
Cada exemplo conecta diretamente os conceitos teóricos aos resultados computacionais, orientando o usuário sobre interpretações e limites operacionais.

\section{Conclusão}
O simulador consolidou teoria e prática de Mecânica dos Fluidos em uma ferramenta acessível, permitindo manipular parâmetros e visualizar consequências físicas instantaneamente. O desenvolvimento reforçou habilidades de Engenharia de Computação ao integrar modelagem matemática, programação interativa, visualização de dados e boas práticas de usabilidade. Foram observadas relações essenciais do medidor de Venturi: dependência de \(C_d\), influência de perdas e regimes de escoamento sobre leituras manométricas. Como trabalhos futuros, recomenda-se incorporar modelos de rugosidade dependentes de Reynolds, suporte a fluidos compressíveis e exportação ampliada (PDF automático, API REST) para integração industrial.

\section{Estilo do Relatório}
Este documento adota linguagem técnica objetiva, apresenta fórmulas em ambiente \texttt{amsmath} quando relevantes, mantém estrutura solicitada (seções 1--7) e integra explicações coesas baseadas no código-fonte e na referência teórica \texttt{Venturi.md}. A organização prioriza clareza, concisão e adequação como documento oficial de entrega.

\end{document}

